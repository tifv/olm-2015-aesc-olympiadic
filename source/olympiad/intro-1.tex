% $date: 2014-09-03

% $authors:
% - Глеб Погудин

% $build$matter[print]: [[.], [.]]

% $style[-no-briefing]:
% - .[no-briefing]
% - verbatim: |-
%     \def\jeolmbriefing{\begingroup
%     \jeolmbriefingtemplate{%
%     Личная письменная олимпиада. Продолжительность~--- 210 минут.\\
%     За~полное решение каждой задачи дается 7 баллов.}
%     \par
%     \iffalse\jeolmbriefingtemplate{%
%     Результаты появятся на~стенде в~холле в~20:00.
%     Желающие смогут посмотреть свои работы в~это~же время в~39~ауд
%     (с~возможностью аппеляции).\\
%     Следующий, устный, тур олимпиады состоится в~СУНЦ МГУ в~это
%     воскресение, 7~сентября.
%     На~него будут приглашены те, кто успешно выступит на~письменном туре
%     (около 30~школьников).}
%     \par
%     \vspace{2ex}
%     \par
%     \hrule
%     \par
%     \jeolmbriefingtemplate{%
%     С~учетом результатов олимпиады будут формироваться группы кружка
%     <<Олимпиадная математика>>, команды на~различные турниры, сборы,
%     и~т.~п.\\
%     Также планируется, что результаты этой олимпиады будет учтены при
%     приглашении на~окружной тур Всероссийской олимпиады.}
%     \par
%     \jeolmbriefingtemplate{%
%     Кружок <<Олимпиадная математика>> будет работать по~средам и~пятницам
%     с~17:00 до~19:00, начиная со~следующей среды (10~сентября).
%     \emph{Следите за~объявлениями рядом с~кабинетом~34.}}%
%     \fi\endgroup}

% $matter[solutions]:
% - verbatim: |-
%     \begingroup \let\ifolympiadsolutions\iftrue
% - .[-solutions]
% - verbatim: \endgroup

% $matter[criteria,-solutions]:
% - verbatim: |-
%     \begingroup \let\ifolympiadcriteria\iftrue
% - .[-criteria]
% - verbatim: \endgroup

% $matter[-no-briefing,-solutions,-criteria]:
% - verbatim: \begingroup\providecommand\jeolmbriefing{}
% - .[no-briefing]
% - verbatim: \jeolmbriefing\endgroup

\begingroup

\providecommand\ifolympiadsolutions{\iffalse}
\providecommand\ifolympiadcriteria{\iffalse}

\newcommand\olympiadsolution[1]{\ifolympiadsolutions#1\fi}
\newcommand\olympiadcriterion[1]{\ifolympiadcriteria#1\fi}

\section*{Сентябрьская олимпиада (письменный тур)}

\olympiadsolution{\subsection*{с решениями}}
\olympiadcriterion{\subsection*{с критериями}}

%\subsection{Простые задачи}

\begin{problems}

%\item
%Сколько существует шестизначных чисел, сумма цифр которых делится на~$5$?
% (простая комби)

\item
Имеется десять коробок, в~каждой лежит по~утюгу.
Разрешена такая операция: выбираются две непустые коробки, из~них вынимается
по~утюгу, а~потом эти два утюга складываются в~некоторую третью коробку
(отличную от~этих двух).
Предложите способ, как при помощи последовательности таких операций можно
собрать все утюги в~одну коробку.

\olympiadsolution{\emph{Решение.}
Будем собирать все утюги в~первую коробку.
Сначала соберем утюги из~коробок №2~и~№3, №4~и~№5, №6~и~№7 в~первую.
Далее, переместим два утюга из~коробок №1~и~№8~в~№2, из~№1~и~№9~в~№3,
из~№1~и~№10 в~№4.
После этого в~коробках №2, №3, и~№4 будет по~2 утюга, а~остальные утюги будут
в~первой.
Наконец, переместим из~коробок №2~и~№3~в~№1, из~№2~и~№4~в~№1,
и~из~№3~и~№4~в~№1.}

\olympiadcriterion{\emph{Критерий:}
$7$ баллов ставится за~любой работающий алгоритм. }

\item
Придумайте натуральное число с~суммой цифр $17$, оканчивающееся на~$17$
и~делящееся на~$17$ одновременно.

\olympiadsolution{\emph{Решение.}
Будем искать число в~виде $\ov{abc17}$, где $\ov{abc}$~--- число, которое
делится на~17, и~имеет сумму цифр $17 - 8 = 9$.
Из~признака делимости на~9 необходимо следует, что $\ov{abc}$ также делится
на~9.
Заметим, что $17 \cdot 9 = 153$ подходит.
\\\emph{Ответ:} $15317$.}

\olympiadcriterion{\emph{Критерий:}
$7$ баллов ставится за~любой правильный ответ
\emph{с~проверкой делимости на~17}, и~5 баллов за~ответ без такой проверки.}

%\item
%Ненулевые действительные числа $a$, $b$ и~$c$ таковы, что
%$a + \frac{1}{b} = 5$, $b + \frac{1}{c} = 12$ и~$c + \frac{1}{a} = 13$.
%Найдите $abc + \frac{1}{abc}$.
% (простая алгебра)

%\item
%Назовем натуральное число \emph{хорошим}, если оно представляется в~виде суммы
%различных степеней тройки.
%Чему равна сумма первых пятнадцати хороших чисел?
%% (простая комби)

%%%%%%%%%%

%\item
%Найдите наименьшее натуральное число, имеющее ровно $28$ делителей.
%% (простая тч)

\item
Решите в~целых числах уравнение $2^x + 3^x + 6^x = 3^y$.
% (простая тч)

\olympiadsolution{\emph{Решение.}
Сначала разберем случай $x \geq 0$.
Заметим, что $3^y > 1$, то~есть $y > 0$.
С~другой стороны, $2^x$ не~может оставаться единственным слагаемым,
не~кратным~3.
Следовательно, $3^x$ и~$6^x$ не~могут быть кратны~3 одновременно.
Это возможно только в~случае $x = 0$, откуда $y = 1$.
\\Разберем теперь случай $x < 0$.
Заметим, что из~$x < 0$ обязательно следует $y \leq 0$.
Домножим уравнение на~$6^{-x} \cdot 3^{-y}$, и~заменим $(-x)$ на~$m$,
а~$(-y)$ на~$n$.
Мы~получаем диофантово уравнение,
$3^n \cdot 3^m + 3^n \cdot 2^m + 3^n = 3^m \cdot 2^m$.
Перепишем это уравнение в~виде $3^n \cdot (3^m + 1) = 2^m (3^m - 3^n)$.
Отсюда следует, во-первых, что $(3^m - 3^n) > 0$, то~есть $m > n$.
С~другой стороны, легко убедиться, что $3^m + 1$ ни~при каких $m$ не~делится
на~8.
Следовательно, и~правая часть не~может делиться на~8, то~есть $2^m < 8$,
откуда $m < 3$.
Так как $m$ положительно, осталось перебрать два случая, $m = 1$ и~$m = 2$.
Первый из~них дает корень $m = 1$, $n = 0$
(соответственно, $x = -1$ и~$y = 0$), а~второй не~дает целых корней.
\\\emph{Ответ:} $x = 0, y = 1$ и~$x = -1, y = 0$.}

\olympiadcriterion{\emph{Критерии:}\begin{enumerate}
\item $0$ баллов~--- угадан только корень $x = 0$, $y = 1$.
\item $1$ балл~--- угаданы оба корня.
\item $3$ балла~--- правильно разобран случай $x \geqslant 0$.
\end{enumerate}}

%%%%%%%%%%

\item\label{/olympiad/intro/4}%
Отрезок~$AB$ имеет длину $4$, точка~$M$~--- его середина.
На~$AB$ как на~диаметре построена окружность~$\omega_1$, на~$AM$~---
окружность~$\omega_2$.
Касательная к~$\omega_2$, проходящая через точку $B$, пересекает $\omega_1$
в~точке~$N$.
Найдите площадь треугольника $AMN$.
% (простая геом)

\olympiadsolution{%
\begin{figure}[htb!]\centering
    \jeolmfigure[width=0.5\textwidth]{4-solution}
    \caption{к~задаче \ref{/olympiad/intro/4}}
    \label{/olympiad/intro/4:solution:fig}
\end{figure}}

\olympiadsolution{\emph{Решение.}
Пусть упомянутая в~условии касательная касается окружности $\omega_2$
в~точке $L$, а~центр этой окружности~--- точка~$C$
(рис.~\ref{/olympiad/intro/4:solution:fig}).
Тогда $BLC$~--- прямоугольный треугольник с~гипотенузой $BC = 3$
и~катетом $LC = 1$, откуда второй катет $BL = \sqrt{8}$.
Далее, треугольник $BNA$ также прямоугольный, так как угол $BNA$ опирается
на~диаметр.
Из~подобия треугольников $BLC$ и~$BNA$ получаем, что
\[
    BL : BN = LC : NA = BC : BA = 3 : 4
,\]
откуда катеты $BN = \frac{4}{3} BL = 8 \sqrt{2} / 3$
и~$NA = \frac{4}{3} LC = 4/3$.
Следовательно, искомая площадь равна
\[
    S(AMN)
=
    \frac{1}{2} \cdot S(BNA)
=
    \frac{1}{2} \cdot \frac{1}{2} \cdot \frac{8 \sqrt{2}}{3} \cdot \frac{4}{3}
=
    \frac{8 \sqrt{2}}{9}
\;.\]
(Первое равенство следует из~того, что медиана $NM$ делит площадь треугольника
$BNA$ пополам.)}

%%%%%%%%%%

%\item
%Пусть $f(x, y) = 3 x^2 + 3 x y + 1$.
%Про числа $a$ и~$b$ известно, что $f(a, b) + 1 = f(b, a) = 42$.
%Найдите $|a + b|$.
%% (простая алгебра)

\end{problems}


%\subsection{Сложные задачи}

\begin{problems}

\item
Рассмотрим произведение $1! \cdot 2! \cdot \ldots \cdot 99! \cdot 100!$.
Какой из~факториалов нужно вычеркнуть, чтобы получился точный квадрат?
Достаточно указать (с доказательством) хотя бы один такой факториал.

\olympiadsolution{\emph{Решение.}
Сгруппируем факториалы по~два подряд идущих.
Заметим, что $(2k - 1)! (2k)! = \left( (2k - 1)! \right)^2 2k$.
Таким образом, все выражение равно квадрату числа
$1! \cdot 3! \ldots \cdot 99!$, умноженному на~$2^{50} 50!$.
$2^{50}$ также является квадратом.
Таким образом, $50!$ можно вычеркнуть.}

\item\label{/olympiad/intro/6}
Общие внешние касательные непересекающихся окружностей $\omega_1$ и~$\omega_2$
касаются их~в~точках $A$ и~$B$, $C$ и~$D$ соответственно
($A$ и~$B$ лежат на одной касательной, $C$ и~$D$ --- на другой).
$M$~--- середина $AB$, отрезки $MC$ и~$MD$ второй раз пересекают соответственно
$\omega_1$ и~$\omega_2$ в~точках $P$ и~$Q$.
Докажите, что точки $A$, $B$, $P$, $Q$ лежат на~одной окружности.

\olympiadsolution{%
\begin{figure}[htb!]\centering
    \jeolmfigure[width=0.5\textwidth]{6-solution}
    \caption{к~задаче \ref{/olympiad/intro/6}}
    \label{/olympiad/intro/6:solution:fig}
\end{figure}}

\olympiadsolution{\emph{Решение.}
Заметим, что
\begin{equation}\label{/olympiad/intro/6:solution:eq:first}
    MC \cdot MP = MA^2 = MB^2 = MD \cdot MQ
.\end{equation}
Рассмотрим точки $X$ и~$Y$, симметричные соответственно точкам $P$ и~$Q$
относительно точки $M$ (рис. \ref{/olympiad/intro/6:solution:fig}).
Тогда из~предыдущего равенства следует
\[
    MA \cdot MB = MC \cdot MX = MD \cdot MY
.\]
Отсюда, так как точки $A$, $B$, $D$, $C$ уже лежат на~одной окружности
(они образуют равнобедренную трапецию), то~и~точки $X$ и~$Y$ обязаны лежать
на~той~же окружности~--- назовем её~$\omega$.
Осталось заметить, что искомая окружность симметрична $\omega$ относительно
точки $M$.}

\olympiadsolution{\emph{Другое решение.}
Из~равенства \ref{/olympiad/intro/6:solution:eq:first} предыдущего решения
можно заключить, что точки $P$ и~$Q$ являются образами точек $C$ и~$D$
соответственно при инверсии относительно точки $M$ с~радиусом $MA$.
Отсюда очевидно, что точки $P$, $Q$, $A$, $B$ лежат на~окружности,
являющейся образом окружности $ABCD$ при той~же инверсии.}

%\item
%В~каждой клетке таблицы $101 \times 101$ написано число $+1$ или $-1$.
%Произведение чисел в~$i$-ой строке обозначим через $R_i$, произведение чисел
%в~$j$-ом столбце~--- через $C_j$.
%Докажите, что число $C_1 + \ldots + C_{101} + R_1 + \ldots + R_{101}$
%не~равно нулю.

\end{problems}


%\subsection{Очень сложные задачи}

\begin{problems}

%\item
%Пусть $n = 2^k - 1$.
%Сколько нечетных коэффициентов %у~многочлена $(x^2 + x + 1)^n$?

\itemx{*}
В~стране четное число городов, некоторые из~которых соединены дорогами.
Из~каждого города выходит четное число дорог и~из~каждого города можно проехать
в~каждый (возможно, с пересадками).
Король хочет закрыть максимально возможное число дорог так, чтобы из~каждого
города все еще можно было проехать в~каждый.
Докажите, что он~может сделать это четным числом способов.

\olympiadsolution{\emph{Решение.}
Переформулируем задачу на~язык теории графов:
в~графе четное число вершин и~все имеют четную степень, нужно доказать, что
число остовных деревьев четно.

Построим новый граф, вершинами которого будут остовные деревья исходного графа.
Вершины, соответствующие деревьям $T$ и~$T'$ будем соединять ребром, если $T$
содержит все ребра $T'$ кроме одного.
Если мы~докажем, что все вершины в~этом графе имеют нечетную степень,
то~мы~покажем, что число остовных деревьев четно.

Докажем~же это.
Выберем в~$T$ ребро~$e$.
Посчитаем, сколько остовных деревьев содержат все ребра $T$ кроме $e$.
После удаления $e$ дерево $T$ распадается на~две компоненты связности.
Обозначим множества вершин в~этих компонентах через $U$ и~$V$.
Найдем количество ребер, соединяющих эти компоненты~--- оно будет на~единицу
больше искомого количества (так как считается ещё ребро $e$, дающее само
дерево~$T$).
Сумма степеней вершин внутри компоненты четна, так как это сумма степеней
вершин в~подграфе.
Однако, сумма степеней (уже учитывая ребра между компонентами) также четна, так
как каждая степень четна.
Стало быть, между компонентами четное число ребер.
Таким образом, каждому ребру~$e$ соответствует нечетное число смежных
с~$T$ деревьев.
Так как вершин в~графе четное количество, в~любом его остовном дереве нечетное
число ребер.
Таким образом, каждая степень в~новом графе получается как сумма нечетного
количества нечетных чисел и, стало быть, нечетна.
Что и~требовалось.}

\end{problems}

\endgroup

