% $date: 2014-09-07

% $authors:
% - Глеб Погудин
% - Илья Воробьев

% $build$style[-0,-1,-print]:
% - .
% - scale_font: [9.8, 11.7]

% $build$matter[0,print]: [[.], [.]]

% $build$matter[1,print]: [[.], [.], [.], [.]]
% $build$style[1,print]:
% - .[tiled4,-print]

% $style[-no-briefing]:
% - .[no-briefing]
% - briefing

% $matter[0]:
% - verbatim: \begingroup \let\ifpreupsurge\iftrue \let\ifupsurge\iffalse
% - .[-0]
% - verbatim: \endgroup

% $matter[1]:
% - verbatim: \begingroup \let\ifpreupsurge\iffalse \let\ifupsurge\iftrue
% - .[-1,no-briefing]
% - verbatim: \endgroup

% $matter[1,briefing]:
% - verbatim: \begingroup \let\ifpreupsurge\iffalse \let\ifupsurge\iftrue
% - .[-1]
% - verbatim: \endgroup

% $matter[solutions,-0,-1]:
% - verbatim: |-
%     \begingroup \let\ifolympiadsolutions\iftrue
% - .[-solutions]
% - verbatim: \endgroup

% $matter[-no-briefing,-solutions,-0,-1]:
% - .[no-briefing]
% - verbatim: \jeolmbriefing

\begingroup

\providecommand\ifpreupsurge{\iftrue}
\providecommand\ifupsurge{\iftrue}

\providecommand\ifolympiadsolutions{\iffalse}

\newcommand\olympiadsolution[1]{\ifolympiadsolutions#1\fi}

\section*{Сентябрьская олимпиада (устный тур)}

\olympiadsolution{\subsection*{с решениями}}

\ifpreupsurge
\subsection*{Довывод}

\begin{problems}

\item
На~утреннике каждый мальчик подарил по~конфете каждой девочке, а~каждая девочка
подарила по~конфете каждому мальчику.
После этого каждый мальчик слопал по~две конфеты, а~каждая девочка скушала
по~три конфеты.
Известно, что общее количество съеденных конфет составляет четверть от~числа
всех подаренных.
Какое наибольшее число детей могло быть на~утреннике?

\olympiadsolution{\emph{Решение.}
<решение>}

\item
Найдите все конечные множества натуральных чисел $S$ такие, что для любых
(не~обязательно различных) $i, j \in S$ число $\frac{i + j}{(i, j)}$ также
принадлежит $S$.
(Через $(i, j)$ обозначается наибольший общий делитель $i$ и~$j$.)
% можно убрать условие "не обязательно различных", тогда задача будет
% содержательнее.

\olympiadsolution{\emph{Решение.}
<решение>}

\item
В~прямоугольном треугольнике $ABC$ с~прямым углом $\angle A$ проведены
биссектрисы $CE$ и~$BD$ (соответственно, $E \in AB$ и~$D \in AC$).
Они пересекаются в~точке~$I$.
Может~ли оказаться, что все из~отрезков $AB$, $AC$, $BI$, $ID$, $CI$ и~$IE$
имеют целую длину?

\olympiadsolution{\emph{Решение.}
<решение>}

%\setproblem{4} % на выбор две комбы
%\itemy{4a}
\item
В~каждой клетке таблицы $101 \times 101$ написано число $+1$ или $-1$.
Произведение чисел в~$i$-ой строке обозначим через $R_i$, произведение чисел
в~$j$-ом столбце~--- через $C_j$.
Докажите, что число $C_1 + \ldots + C_{101} + R_1 + \ldots + R_{101}$
не~равно нулю.

\olympiadsolution{\emph{Решение.}
<решение>}

%\itemy{4b}
%\item
%Глеб и~Юлик играют на~бесконечной сетке из~правильных шестиугольников.
%Исходно все шестиугольники пусты.
%Первым ходит Глеб, своим ходом он~выбирает два смежных шестиугольника и~кладет
%туда по~одному камню.
%За~свой ход Юлик может убрать один камень из~любого шестиугольника.
%Глеб выиграет, если сможет выстроить $k$~камней в~ряд.
%При каком наибольшем $k$ Глеб гарантированно выиграет?

\item
Про многочлен $p(x)$ степени $n$ известно, что $p(k) = 1 / k$
для всех $k = 1, \ldots, n + 1$.
Найдите $p(n + 2)$.

\olympiadsolution{\emph{Решение.}
<решение>}

\end{problems}
\fi % \ifpreupsurge


\ifupsurge
\subsection*{Вывод}
\setproblem{5}

\begin{problems}

\item
Пусть натуральные числа $p$, $q$ и~$r$ попарно взаимно просты.
Рассмотрим прямоугольный параллелепипед размера $p \times q \times r$.
Обозначим через $A$ одну из~его вершин, а~через $B$, $C$ и~$D$ вершины смежные
с~ней.
Через точки $B$, $C$ и~$D$ проведена плоскость.
Сколько единичных кубиков она пересекает?
% от Ильи Воробьева

\olympiadsolution{\emph{Решение.}
<решение>}

\item
В~трапеции $ABCD$ ($AD \parallel BC$) углы $C$ и~$D$ прямые.
Окружность $\omega$, построенная на~$AB$ как на~диаметре, пересекает отрезок
$AD$ второй раз в~точке $P$.
Касательная к~$\omega$ в~точке $P$ пересекает прямую $CD$ в~точке $M$.
Вторая касательная из~$M$ к~$\omega$ касается $\omega$ в~точке $Q$.
Докажите, что прямая $BQ$ проходит через середину $CD$.

\olympiadsolution{\emph{Решение.}
<решение>}

\item
Даны натуральные $m$ и~$n$.
Докажите, что существует натуральное~$c$ такое, что каждая ненулевая цифра
входит в~десятичные записи $c \cdot m$ и~$c \cdot n$ одно и~то~же число раз.

\olympiadsolution{\emph{Решение.}
<решение>}

\end{problems}
\fi % \ifupsurge

\endgroup

