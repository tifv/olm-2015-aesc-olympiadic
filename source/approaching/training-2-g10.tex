% $date: 2014-12-05

\section*{Тренировочная олимпиада --- 2, 10 класс}

% $build$matter[print]: [[.], [.]]

\begin{problems}

\item
Даны $n + 1$ попарно различных натуральных чисел, меньших $2 n$ ($n > 1$).
Доказать, что среди них найдутся три таких числа, что сумма двух из~них равна
третьему.

\item
Найдите все такие пары чисел $(p, q)$, что каждое из~уравнений
$x^2 - px + q = 0$  и~$x^2 - qx + p = 0$ имеет два различных натуральных корня.

\item
Есть 100 коробок, пронумерованных числами от~1 до~100.
В~одной коробке лежит приз и~ведущий знает, где он~находится.
Зритель может послать ведущему пачку записок с~вопросами, требующими ответа
<<да>> или <<нет>>.
Ведущий перемешивает записки в~пачке и, не~оглашая вслух вопросов, честно
отвечает на~все.
Какое наименьшее количество записок нужно послать, чтобы наверняка узнать,
где находится приз?

\item
В~окружности с~центром~$O$ проведена хорда~$AB$ и~радиус~$OK$, пересекающий
её~под прямым углом в~точке~$M$.
На~большей дуге~$AB$ окружности выбрана произвольная точка~$P$, отличная
от~середины этой дуги.
Прямая~$PM$ вторично пересекает окружность в~точке~$Q$, а~прямая~$PK$
пересекает $AB$ в~точке~$R$.
Докажите, что $KR > MQ$.

\end{problems}

\begin{center}\small\sffamily
Окружной тур в~это воскресенье, 7 декабря.
Не~продолбайте!
\end{center}

