% $date: 2015-01-28

\section*{Алгебраический разнобой --- добавка}

% $authors:
% - Юлий Тихонов

% $build$matter[print]: [[.], [.], [.], [.]]
% $build$style[print]:
% - .[tiled4,-print]

% $matter[-contained,no-header]:
% - verbatim: \setproblem{10}
% - .[contained]

% $matter[contained,-no-header]:
% - .[no-header]

\begin{problems}

\item
Дано натуральное число $n$ такое, что число $2^n + 1$~--- простое.
Докажите, что для некоторого целого $k$ выполнено $n = 2^k$.

\item
На~доске выписана десятичная запись числа $7^{2013}$.
Вася осуществляет следующий процесс: он~стирает первую цифру написанного
на~доске числа и~прибавляет её~к~оставшемуся числу, после чего выписывает
результат на~доску вместо старого числа.
Когда на~доске, наконец, появилось десятизначное число, Вася успокоился.
Докажите, что какие-то~две цифры этого числа совпадают.

\item
Для каждого натурального $n$ обозначим через $S_n$ сумму первых $n$ простых
чисел: $S_1 = 2$, $S_2 = 2+3 = 5$, $S_3 = 2 + 3 + 5 = 10$, $\ldots$
Могут~ли два подряд идущих члена последовательности $(S_n)$ оказаться
квадратами натуральных чисел?
\emph{\small Регион 2010, 9.8}

\item
Дан выпуклый пятиугольник.
Петя выписал в~тетрадь значения синусов всех его углов, а~Вася~--- значения
косинусов всех его углов.
Оказалось, что среди выписанных Петей чисел нет четырех различных.
Могут~ли все числа, выписанные Васей, оказаться различными?
\emph{\small Регион 2012, 10.5}

\item
Натуральное число~$m$ таково, что сумма цифр в~десятичной записи числа~$2^m$
равна 8.
Может~ли при этом последняя цифра числа $2^m$ быть равной 6?
\emph{\small Регион 2009, 10.5}

\end{problems}

