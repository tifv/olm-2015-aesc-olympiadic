% $date: 2014-10-31

\section*{Китайская теорема об остатках}

% $authors:
% - Виктор Трещёв

% $build$matter[print]: [[.], [.]]

\begin{problems}

\item\textbf{Китайская теорема об остатках.}
Пусть имеется два набора чисел $a_1, a_2, \ldots, a_n$
и~$m_1, m_2, \dots, m_n$, причем числа во~втором наборе попарно взаимно просты.
Тогда система сравнений
\[ \left\{ \begin{aligned} &
    x \equiv a_1 \pmod {m_1};
\\ &
    x \equiv a_2 \pmod {m_2};
\\ &
    \dots
\\ &
    x \equiv a_n \pmod {m_n};
\end{aligned} \right. \]
имеет единственное решение по~модулю $m_1 \cdot m_2 \cdot \ldots \cdot m_n$.

\textbf{Подсказка для комбинаторного доказательства.}
Сколько существует различных остатков по~модулю $m_1$?
А~сколько существует пар остатков таких, что первый из~них $\mod m_1$,
а~второй $\mod m_2$?
А~сколько существует различных наборов остатков таких, что первый из~них
$\mod m_1$, второй $\mod m_2$, \ldots, $n$-й~--- $\mod m_n$?

\textbf{Подсказка для конструктивного доказательства.}
Попробуйте придумать решение для системы, в~которой один остаток $a_i$
равен $1$, а~все остальные $a_j, j \neq i$, равны $0$.
При помощи таких решений (для разных $i$) выписывается решение для произвольной
системы.

\item
При каких целых~$n$ число $a_n = n^2 + 3n + 1$ делится на~$55$?

\item
При каких целых $x$ выполняется данная система сравнений?
\[ \left\{ \begin{aligned} &
    x \equiv 3 \pmod{5};
\\ &
    x \equiv 7 \pmod{17}.
\end{aligned} \right. \]

\item
\sp
Трехзначное число $625$ обладает своеобразным свойством самовоспроизводимости,
как то: $625^2 = 390 625$.
Сколько четырехзначных чисел удовлетворяют уравнению
$x^2 \equiv x \pmod{10000}$?
\\
\sp
Докажите, что при любом $k$ существует ровно $4$ набора из~$k$ цифр~---
$00\ldots00$, $00\ldots01$ и~еще два, оканчивающиеся пятеркой и~шестеркой,~---
обладающие таким свойством: если натуральное число оканчивается одним из~этих
наборов цифр, то~его квадрат оканчивается тем~же набором цифр.

\item\textbf{Больное войско.}
Генерал хочет построить для парада своих солдат в~одинаковые квадратные каре,
но~он~не~знает, сколько солдат (от~$1$ до~$37$) находится в~лазарете.
Докажите, что у~генерала может быть такое количество солдат, что он, независимо
от~заполнения лазарета, сумеет выполнить свое намерение.
Например, войско из~$9$ человек можно поставить в~виде квадрата $3 \times 3$,
а~если один человек болен, то~в~виде двух квадратов $2 \times 2$.
(Каре $1 \times 1$, естественно, не~допускается.)

\end{problems}

