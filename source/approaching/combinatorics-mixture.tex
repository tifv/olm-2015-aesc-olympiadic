% $date: 2015-01-23

\section*{Комбинация задач}

% $authors:
% - Виктор Трещёв

% $matter[full-version]:
% - - .[-full-version]
% - - ../combinatorics-mixture-more

\begin{problems}

\item
Фокусник выкладывает $36$ карт в~виде квадрата $6\times 6$ (в~$6$~столбцов
по~$6$~карт) и~просит Зрителя мысленно выбрать карту и~запомнить столбец,
её~содержащий.
После этого Фокусник определенным образом собирает карты, снова выкладывает
в~виде квадрата $6\times6$ и~просит Зрителя назвать номера столбцов, содержащих
выбранную карту в~первый и~второй раз.
После ответа Зрителя Фокусник безошибочно отгадывает карту.
Как действовать Фокуснику, чтобы фокус гарантированно удался?

\item
Фигура <<мамонт>> бьет как слон (по~диагоналям), но~только в~трех направлениях
из~четырех
(отсутствующее направление может быть разным для разных мамонтов).
Какое наибольшее число не~бьющих друг друга мамонтов можно расставить
на~шахматной доске $8 \times 8$?

\item
Дан квадрат $n \times n$.
Изначально его клетки раскрашены в~белый и~черный цвета в~шахматном порядке,
причем хотя~бы одна из~угловых клеток черная.
За~один ход разрешается в~некотором квадрате $2\times 2$ одновременно
перекрасить входящие в~него четыре клетки по~следующему правилу: каждую белую
перекрасить в~черный цвет, каждую черную – в~зеленый, а~каждую зеленую~---
в~белый.
При каких $n$ за~несколько ходов можно получить шахматную раскраску, в~которой
черный и~белый цвета поменялись местами?

\item
На~окружности отмечено $2 N$ точек ($N$~--- натуральное число).
Известно, что через любую точку внутри окружности проходит не~более двух хорд
с~концами в~отмеченных точках.
Назовем паросочетанием такой набор из~$N$~хорд с~концами в~отмеченных точках,
что каждая отмеченная точка является концом ровно одной из~этих хорд.
Назовем паросочетание \emph{четным}, если количество точек, в~которых
пересекаются его хорды, четно, и~нечетным иначе.
Найдите разность между количеством четных и~нечетных паросочетаний.

\item
Главная аудитория фирмы <<Рога и~копыта>> представляет собой квадратный зал
из~восьми рядов по~восемь мест.
$64$~сотрудника фирмы писали в~этой аудитории тест, в~котором было шесть
вопросов с~двумя вариантами ответа на~каждый.
Могло~ли так оказаться, что среди наборов ответов сотрудников нет одинаковых,
причем наборы ответов любых двух людей за~соседними столами совпали не~больше,
чем в~одном вопросе?
(Столы называются соседними, если они стоят рядом в~одном ряду или друг
за~другом в~соседних рядах.)

\item
$2011$ складов соединены дорогами так, что от~каждого склада можно
проехать к~любому другому, возможно, проехав по~нескольким дорогам.
На~складах находится по~$x_1, \ldots, x_{2011}$ кг цемента соответственно.
За~один рейс можно провезти с~произвольного склада на~другой по~соединяющей
их~дороге произвольное количество цемента.
В~итоге на~складах по~плану должно оказаться по~$y_1, \ldots, y_{2011}$~кг
цемента соответственно, причем
$x_1 + x_2 + \ldots + x_{2011} = y_1 + y_2 + \ldots + y_{2011}$.
За~какое минимальное количество рейсов можно выполнить план при любых значениях
чисел $x_i$ и~$y_i$ и~любой схеме дорог?

\end{problems}

