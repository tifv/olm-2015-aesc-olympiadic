% $date: 2015-02-27

\section*{Тригонометрия такая тригонометрия}

% $authors:
% - Виктор Трещёв

% $build$matter[print]: [[.], [.], [.], [.]]
% $build$style[print]:
% - .[tiled4,-print]

\begin{problems}

\item
Упростите выражение:
\(
    \cos (a) \cdot \cos (2 a) \cdot \cos (4 a)
    \cdot \ldots \cdot
    \cos (2^{n - 1} a)
\).

\item
Для каких значений x выполняется неравенство
$2^{\sin^2 (x)} + 2^{\cos^2 (x)} \geq 2 \sqrt{2}$?

\item
Решите уравнение $\sin (x) + \sin (2 x) + \sin (3 x) = 0$.

\item
Докажите, что при $0 \leq \phi \leq \frac{\pi}{2}$ выполняется неравенство
$\cos \bigl( \sin (\phi) \bigr) > \sin \bigl( \cos (\phi) \bigr)$.

\item
Пусть $A$~--- произвольный угол, $B$ и~$C$~--- острые углы.
Всегда~ли существует такой угол $X$, что
\[
    \sin (X)
=
    \frac{
        \sin (B) \cdot \sin (C)
    }{
        1 - \cos (A) \cdot \cos (B) \cdot \cos (C)}
\;?\]

\item
Докажите равенство:
$\tg (20^\circ) \cdot \tg (40^\circ) \cdot \tg (80^\circ) = \sqrt{3}$.

\item
Докажите равенство:
\[
    \cos {\frac{\pi}{15}} \cdot
    \cos {\frac{2 \pi}{15}} \cdot
    \cos {\frac{3 \pi}{15}} \cdot
    \cos {\frac{4 \pi}{15}} \cdot
    \cos {\frac{5 \pi}{15}} \cdot
    \cos {\frac{6 \pi}{15}} \cdot
    \cos {\frac{7 \pi}{15}}
=
    \left( \frac{1}{2} \right)^{7}
\;.\]

\item
Даны различные натуральные числа $a$ и~$b$.
На~координатной плоскости нарисованы графики функций $y = \sin (a x)$
и~$y = \sin (b x)$ и~отмечены все точки их~пересечения.
Докажите, что существует натуральное число $c$, отличное от~$a$ и~$b$ и~такое,
что график функции $y = \sin (c x)$ тоже проходит через все отмеченные точки.

\end{problems}

