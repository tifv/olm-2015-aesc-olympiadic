% $date: 2014-09-17

\section*{Ряды Фарея}

% $authors:
% - Юлий Тихонов

% $build$matter[print]: [[.], [.]]

{\def\thefootnote{$^\circ$}%
\footnotetext[0]{Этим символом отмечены задачи и пункты, являвшиеся частью
лекции. Сдавать их нельзя.}}

\definition
\emph{Ряд Фарея} $F_n$~---
это последовательность всех рациональных чисел на $[0; 1]$ со знаменателем не
больше $n$, упорядоченных по возрастанию.
\begin{gather*}
    F_1
=
    \left\{
        \tfrac{0}{1}, \, \tfrac{1}{1}
    \right\}
,\quad
    F_2
=
    \left\{
        \tfrac{0}{1}, \, \tfrac{1}{2}, \, \tfrac{1}{1}
    \right\}
,\quad
    F_3
=
    \left\{
        \tfrac{0}{1}, \, \tfrac{1}{3}, \, \tfrac{1}{2}, \,
        \tfrac{2}{3}, \, \tfrac{1}{1}
    \right\}
,\\
    F_4
=
    \left\{
        \tfrac{0}{1}, \, \tfrac{1}{4}, \, \tfrac{1}{3}, \,
        \tfrac{1}{2}, \, \tfrac{2}{3}, \, \tfrac{3}{4}, \,
        \tfrac{1}{1}
    \right\}
,\quad
    F_5
=
    \left\{
        \tfrac{0}{1}, \, \tfrac{1}{5}, \, \tfrac{1}{4}, \,
        \tfrac{1}{3}, \, \tfrac{2}{5}, \, \tfrac{1}{2}, \,
        \tfrac{3}{5}, \, \tfrac{2}{3}, \, \tfrac{3}{4}, \,
        \tfrac{4}{5}, \, \tfrac{1}{1}
    \right\}
.\end{gather*}

\observation
Фраза <<дробь $\frac{x}{y}$ в ряду Фарея $F_n$>> подразумевает, что дробь
несократимая.

\begin{problems}

\itemx{$^\circ$}
Докажите, что две различные несократимые дроби $\tfrac{a}{n}$ и~$\tfrac{b}{n}$
никогда не~могут стоять рядом в~ряду Фарея
(не~считая $\tfrac{0}{1}$ и $\tfrac{1}{1}$).

\end{problems}

\definition
Дробь $\frac{a + c}{b + d}$ называется \emph{медиантой}
дробей $\frac{a}{b}$ и $\frac{c}{d}$.

\begin{problems}

\item
Докажите, что медианта двух дробей всегда находится между ними.
(Рассматриваются только положительные дроби.)

\itemx{$^\circ$}
Пусть две дроби $\tfrac{a}{b}$ и $\frac{c}{d}$ стоят рядом в ряду Фарея $F_n$.
Докажите, что $b + d > n$.

\itemx{$^\circ$}
Докажите, что расстояние между дробями $\frac{a}{b}$ и $\frac{c}{d}$ не~меньше
$\frac{1}{bd}$.

\item
\spx{$^\circ$}
Пусть дробь $\frac{x}{n}$ располагается в ряду Фарея $F_n$ между $\frac{a}{b}$
и $\frac{c}{d}$.
Докажите, что
\[
    \frac{c}{d} - \frac{a}{b} = \frac{1}{bd}
\quad\Rightarrow\quad
    \frac{x}{y} - \frac{a}{b} = \frac{1}{by}
\text{\quadи\quad}
    \frac{c}{d} - \frac{x}{y} = \frac{1}{yd}
\]
\spx{$^\circ$}
Докажите, что для любых двух соседних дробей $\frac{a}{b}$ и $\frac{c}{d}$
в ряду Фарея выполнено $b c - a d = 1$.
\\
\sp
Пусть $\frac{a}{b}$ и $\frac{c}{d}$~--- две дроби между $0$ и $1$, причем
$b c - a d = 1$.
Докажите, что в некотором ряду Фарея эти дроби соседние.

\item
Каково\quad
\sp максимальное\quad
и\quad
\sp минимальное\quad
расстояние между соседними дробями в $F_n$?

% утащено у Илюхиной Марии
\item
Все дроби со знаменателями $p$, $q$ и $p + q$
($p$ и $q$ взаимно просты, дроби не обязательно несократимые)
упорядочили по возрастанию.
Докажите, что все знаменатели на нечетных местах равны $p + q$.

\end{problems}

