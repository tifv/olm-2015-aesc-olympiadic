% $date: 2014-09-24

\section*{Многоугольники на решётке}

% $authors:
% - Юлий Тихонов

% $build$matter[print]: [[.], [.], [.], [.]]
% $build$style[print]:
% - .[tiled4,-print]

\begin{problems}

\item
\sp
Докажите, что в~любом многоугольнике можно провести либо внутреннюю,
либо внешнюю диагональ.
\\
\spx{*}
Докажите, что в любом многоугольнике можно провести внутреннюю диагональ.

\end{problems}

\definition
\emph{Многоугольник на целочисленной решётке}~--- это многоугольник, все
вершины которого лежат в точках с~целыми координатами.

Далее все многоугольники~--- это многоугольники на решётке.

\begin{problems}

\item
Докажите, что любой многоугольник, если он более-чем-три-угольник, можно
представить как объединение либо разность двух многоугольников
с меньшим количеством вершин.

\end{problems}

\definition
\emph{Примитивный треугольник}~--- треугольник с вершинами на решётке, внутри
и~на~границе которого нет точек решётки, кроме вершин этого треугольника.

\begin{problems}

\item
\emph{Пункты сдаются одновременно.}
\\
\sp
Докажите, что если в треугольнике есть точка решётки
(т.~е. точка с~целочисленными координатами),
то этот треугольник можно представить как объединение трех треугольников.
\\
\sp
Докажите, что если на границе треугольника есть точка решётки,
то этот треугольник можно представить как объединение двух треугольников.

\itemx{$^\circ$}
Площадь любого примитивного треугольника равна $1/2$.

\end{problems}

\claim{Формула Пика}
Пусть многоугольник с вершинами на решетке содержит $n$ внутренних точек решётки
и $m$ точек решётки на своей границе (включая вершины).
Тогда его площадь равна
\[
    S = n + m/2 - 1
\]

\begin{problems}

\itemx{$^\circ$}
Проверьте формулу Пика для примитивных треугольников.

\item
\emph{Пункты сдаются одновременно.}
\\
\sp
Пусть многоугольник $A$ представляется в виде объединения двух многоугольников
$B$ и $C$.
Докажите, что если формула Пика верна для $B$ и $C$, то она верна и для $A$.
\\
\sp
То же самое, если $A$ представляется в виде разности двух многоугольников.
\\
\sp
Докажите формулу Пика для всех многоугольников на решётке.

\end{problems}

