% $date: 2015-02-13

\section*{Функциональные уравнения}

% $authors:
% - Тихонов Юлий
% - Виктор Трещёв

% $build$matter[print]: [[.], [.], [.], [.]]
% $build$style[print]:
% - .[tiled4,-print]

% $build$matter[full-version,print]: [[.], [.]]
% $build$style[print]:
% - .

% $matter[full-version,-no-header]:
% - .[header,multidate]
% $matter[full-version]:
% - .[-full-version]
% - ../functional-equation-more[contained]

Найдите все функции $f$, удовлетворяющие уравнению (если не указано иное).

\begin{problems}

\item
$2f(x) + f(1-x) = x^2$, где
$f \colon \mathbb{R} \to \mathbb{R}$.

\item
$f(x) + 2f(1/x) = 3x$, где
$f \colon \mathbb{R} \setminus \{0\} \to \mathbb{R}$.

\item
$f(x + y) + f(x - y) = 2f(x)\cos y$, где
$f \colon \mathbb{R} \to \mathbb{R}$.

\item
Существуют ли функции
$f, g \colon \mathbb{R} \to \mathbb{R}$
такие, что при любых $x$
\\
\sp
$f(g(x)) = x^2$, $g(f(x)) = x^3$?
\quad\sp
$f(g(x)) = x^2$, $g(f(x)) = x^4$?

\item
Найдите все
$f \colon \mathbb{Q} \to \mathbb{Q}$,
такиеm что
$f(1) = 2$ и
$f(x y) = f(x) f(y) - f(x + y) + 1$
при всех $x, y \in \mathbb{Q}$.

\item
Дано натуральное число $n$.
Найдите все непрерывные $f \colon [0, 1] \to \mathbb{R}$ такие, что
$f(0) = 0$, $f(1) = 1$,
\(
    \underbrace{
        f(f(\ldots\,f
    }_{n}
    (x)\ldots))
=
    x
\)
для любого $x \in [0,1]$.

\item
Найти все непрерывные $f \colon \mathbb{R} \to \mathbb{R}$, такие что
\(
    f(x + y)
=
    f(x) + f(y) + x y (x + y)
\).

\item
Найти все функции $f \colon \mathbb{Q} \to \{-1, 1\}$ такие, что равенство
$f(x) \cdot f(y) = -1$
выполнено для всех $x \neq y \in \mathbb{Q}$ при любом из условий
$x + y = 0$, $x + y = 1$, $x \cdot y = 1$.

\item
Дано $n \in \mathbb{N}$.
Определенная для точек плоскости вещественнозначная функция такова, что
$f(A_1) + f(A_2) + \ldots + f(A_n) = 0$ выполняется для любого правильного
$n$-угольника $A_1 A_2 \ldots A_n$.
Найдите все такие функции.

\itemx{*}
Докажите, что функция $f \colon \mathbb{R} \to \mathbb{R}$, удовлетворяющая одному из
следующих двух тождеств:
\(
    f(x) + f(y)
\equiv
    f(x + y)
\),
\(
    f(x \cdot y + x + y)
\equiv
    f(x \cdot y) + f(x) + f(y)
\),
удовлетворяет и другому.

\end{problems}

