% $date: 2014-09-26

\section*{Инварианты}

% $authors:
% - Виктор Трещёв

% $build$matter[print]: [[.], [.]]

\begin{problems}

\item
На~вешалке висят 20~платков.
17 девочек по~очереди подходят к~вешалке, и~каждая либо снимает, либо вешает
ровно один платок.
Может~ли после ухода девочек на~вешалке остаться 10~платков?

\item
В~таблице $m \times n$ расставлены числа так, что сумма чисел в~любой строке
или столбце равна~1.
Докажите, что $m = n$.

\item
В~одном бидоне находится 1\,л воды, а~в~другом~--- 1\,л спирта.
Разрешается переливать любую часть жидкости из~одного бидона в~другой.
Можно~ли добиться, чтобы во~втором бидоне концентрация спирта оказалась меньше
50\%?

\item
В~одной клетке квадратной таблицы $4 \times 4$ стоит знак минус, а~в~остальных
стоят плюсы.
Разрешается одновременно менять знак во~всех клетках, расположенных в~одной
строке или в~одном столбце.
Докажите, что, сколько~бы мы~ни~проводили таких перемен знака, нам не~удастся
получить таблицу из~одних плюсов.

\item
На~44~деревьях, расположенных по~окружности, сидели 44 веселых чижа
(на~каждом дереве по~чижу).
Время от~времени два чижа одновременно перелетают на~соседние деревья в~разных
направлениях (один~--- по~часовой стрелке, другой~--- против).
Докажите, что чижи не~смогут собраться на~одном дереве.

\item
Круг разделен на~6~секторов, в~котором по~часовой стрелке стоят числа
\[1, 0, 1, 0, 0, 0.\]
Можно прибавлять по~единице к~любым числам, стоящим в~двух соседних секторах.
Можно~ли сделать все числа равными?

\item
Можно~ли круг разрезать на~несколько частей, из~которых сложить квадрат?
(Разрезы~--- это участки прямых и~дуги окружностей.)

\item
На~горе 1001~ступенька, на~некоторых ступеньках лежат камни, по~одному
на~ступеньке.
Сизиф берет любой камень и~переносит его вверх на~ближайшую свободную ступеньку
(т.~е. если ближайшая ступенька свободна, то~на~неё, а~если она занята,
то~на~несколько ступенек вверх до~первой свободной).
После этого Аид скатывает на~одну ступеньку вниз один из~камней, у~которых
предыдущая ступенька свободна.
Камней 500 и~первоначально они лежали на~нижних 500 ступеньках.
Сизиф и~Аид действуют по~очереди, начинает Сизиф.
Цель Сизифа~--- положить камень на~верхнюю ступеньку.
Может~ли Аид ему помешать?

\end{problems}

