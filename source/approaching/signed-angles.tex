% $date: 2015-02-18

\section*{Ориентированные углы}

% $authors:
% - Алексей Доледенок
% - Юлий Тихонов

% $matter[full-version,-no-header]:
% - .[header]
% $matter[full-version]:
% - - .[-full-version]
% - - ../simson-line[contained]

% $build$matter[print]: [[.], [.]]

\definition
Ориентированным углом между прямыми $l$ и $m$ называется такой угол, на который
нужно против часовой стрелки повернуть прямую $l$, чтобы она стала параллельна
прямой $m$.
Обозначается ориентированный угол через $\angle(l, m)$.
Углы, отличающиеся на кратное 180 число градусов, считаются равными.

\begin{problems}

\itemx{$^\circ$}
Свойства ориентированных углов:
\begin{enumerate}
	\item $\angle (l, m) = - \angle (m, l)$.
	\item $\angle (l, m) + \angle (m, k) = \angle (l, k)$.
	\item \(
        \angle (AC, CB) = \angle (AD, DB)
    \Leftrightarrow
        \text{точки $A$, $B$, $C$ и $D$ на одной окружности.}
    \)
    % Заметим, что если считать, что прямая $AA$ это касательная в точке $A$
    % к окружности, то этот факт верен ВСЕГДА и никак не зависит
    % от расположения точек на окружности.
	\item $\angle (l,m) = \angle (l,k) \Leftrightarrow m \parallel k$.
\end{enumerate}

\itemx{$^\circ$}
Две окружности пересекаются в точках $P$ и $Q$.
Через $P$ проходит прямая $AB$, причем $A$ лежит на первой окружности,
а $B$~--- на второй.
Через $Q$ проходит прямая $CD$, причем $C$ лежит на первой окружности,
а $D$~--- на второй.
Докажите, что $AC \parallel BD$.

\end{problems}

\emph{Докажите через ориентированные углы.}

\begin{problems}

\item
Даны окружности $S_1$, $S_2$ и $S_3$, проходящие через точку $X$.
Вторая точка пересечения окружностей $S_1$ и $S_2$~--- точка $P$,
$S_2$ и $S_3$~--- точка $Q$, $S_3$ и $S_1$~--- точка $R$.
На окружности $S_1$ выбрана произвольная точка $A$.
Вторая точка пересечения прямой $AP$ с $S_2$~--- точка $B$,
прямой $AR$ с $S_3$~--- точка $C$.
Докажите, что $B$, $C$ и $Q$ лежат на одной прямой.

\item
На~окружности даны точки $A$, $B$, $C$, $D$.
$M$~---середина дуги~$AB$.
Обозначим точки пересечения хорд $MC$ и~$MD$ с~хордой~$AB$ через~$E$ и~$K$.
Докажите, что точки $K$, $E$, $C$ и~$D$ лежат на~одной окружности.

\item
Даны 4 прямые общего положения.
Всеми возможными способами выкидывается одна из~них, и~берется описанная
окружность оставшегося треугольника.
Докажите, что четыре таких окружности проходят через одну точку.
Эта точка называется \emph{точкой Микеля} для этой четвёрки прямых
(или для четырёхугольника, образованного этими прямыми).

\end{problems}

