% $date: 2015-01-30

\section*{Комбинация задач (продолжение)}

% $authors:
% - Павел Кожевников
% - Виктор Трещёв

\begin{problems}

\item 
Поезд двигался по направлению из $A$ в $B$ в течение $3.5$ часов.
Известно, что во время движения за любой промежуток времени в $1$ час поезд
проходит ровно $60\,\text{км}$.
Может ли расстояние между $A$ и $B$ равняться
\quad
\sp
$200\,\text{км}$;
\quad
\sp
$250\,\text{км}$?

\item 
В клетках квадрата
\quad
\sp $3 \times 3$;
\quad
\sp $7 \times 7$
\quad
расставлены числа.
Известно, что сумма чисел в клетках любого трехклеточного уголка положительна.
Верно ли, что сумма всех чисел положительна?

\item 
В ряд расставляют $2 n$ сапог: $n$ левых и $n$ правых.
Всегда ли можно выбрать 10 идущих подряд сапог так, чтобы среди них было
поровну левых и правых, если
\quad
\sp $n = 15$;
\quad
\sp $n = 12$?

\item 
На доске написано число 12.
В течение каждой минуты число либо умножают, либо делят на 2 или на 3, и
результат записывают на доску вместо исходного числа.
Докажите, что число, которое будет написано на доске ровно через час, не может
быть равно 54.

\item 
На плоскости расположено 2013 красных, 2013 белых и 2013 синих точек, никакие
три из которых не лежат на одной прямой.
Некоторые пары разноцветных точек соединены отрезками, причем из каждой точки
выходит одинаковое число отрезков.
Докажите, что найдется красная точка, которая соединена и с белой, и с синей
точками.

\item 
Изначально клетки доски $8 \times 8$ окрашены в шахматном порядке.
За одну операцию разрешено перекрасить клетки в квадратике $2 \times 2$.
Можно ли за несколько операций получить ситуацию, в которой покрашены черным
\quad
\sp все клетки, кроме одной;
\quad
\sp в точности все клетки одной главной диагонали?

\item 
Существует ли на плоскости замкнутся 2013-звенная ломаная, у которой все вершины
имеют целые координаты, а все звенья имеют одинаковую длину?

\item 
На 16 карточках написаны натуральные числа от 1 до 16.
За один вопрос можно указать любое подмножество карточек и узнать множество
чисел, записанных на этих карточках.
За какое наименьшее число вопросов можно узнать число на каждой
карточке?

\item 
Табло состоит из нескольких лампочек, каждая из которых может гореть или не
гореть.
На пульте несколько кнопок, каждая кнопка изменяет состояние определенного
множества лампочек.
Могло ли оказаться, что на табло можно сделать ровно 100 различных узоров?

\item 
Имеется $11$ красных и $11$ желтых карточек.
Карты каждого цвета помечены числами $2^0$, $2^1$, $\ldots$, $2^{10}$.
Группа карт называется \emph{хорошей}, если сумма чисел на картах этой
группы равна 2013.
Найдите количество хороших групп карт.

\end{problems}

