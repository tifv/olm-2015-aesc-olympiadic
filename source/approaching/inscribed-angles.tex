% $date: 2014-10-29

\section*{Вписанные углы}

% $authors:
% - Фёдор Ивлев
% - Юлий Тихонов

% $build$matter[print]: [[.], [.], [.], [.]]
% $build$style[print]:
% - .[tiled4,-print]

\begin{problems}

\item
Две окружности пересекаются в точках $P$ и $Q$.
Через $P$ и $Q$ проведены прямые $AB$ и $CD$, пересекающие первую окружность
в~точках $A$ и $C$, а вторую~--- в~точках $B$ и $D$.
Докажите, что $AC \parallel BD$.

\item
Вершина $A$ остроугольного треугольника $ABC$ соединена отрезком с центром~$O$
описанной окружности.
Из вершины~$A$ проведена высота~$AH$.
Докажите, что~$\angle BAH = \angle OAC$.

\item
Две окружности пересекаются в~точках~$P$ и~$Q$.
Третья окружность с центром~$P$ пересекает первую окружность в~точках~$A$
и~$B$, а вторую~--- в~точках~$C$ и~$D$.
Докажите, что $\angle AQD = \angle BQC$.

\item
Известно, что в~некотором треугольнике медиана, биссектриса и~высота,
проведенные из~вершины~$C$, делят угол на~четыре равные части.
Найдите углы этого треугольника.

\end{problems}

\statement
Дан вписанный четырехугольник $ABCD$.
Его диагонали пересекаются в~точке~$M$, а лучи~$AB$ и~$DC$~--- в~точке~$P$.
Тогда угол $AMB$ равен полусумме дуг $AB$ и~$CD$,
а угол $BPC$ равен полуразности дуг $AD$ и~$BC$.

\begin{problems}

\item
Точка $O$, лежащая внутри треугольника $ABC$, обладает тем свойством, что
прямые~$AO$, $BO$ и~$CO$ проходят через центры описанных окружностей
треугольников~$BCO$, $ACO$ и~$ABO$.
Докажите, что $O$~--- центр вписанной окружности треугольника~$ABC$.

\item\emph{Лемма о трезубце.}
Продолжение биссектрисы угла $B$ треугольника~$ABC$ пересекает описанную
окружность в~точке~$M$; $I$~--- центр вписанной окружности, $I_b$~--- центр
вневписанной окружности, касающейся стороны~$AC$.
Докажите, что точки~$A$, $C$, $I$ и~$I_b$ лежат на окружности с~центром~$M$.

\item
На окружности даны точки $A$, $B$, $C$, $D$ в~указанном порядке.
Точка~$M$~--- середина дуги~$AB$.
Обозначим точки пересечения хорд~$MC$ и~$MD$ с~хордой~$AB$ через~$E$ и~$K$.
Докажите, что $KECD$~--- вписанный четырехугольник.

\item
Пятиугольник $ABCDE$, все углы которого тупые, вписан в~окружность $\omega$.
Продолжения сторон $AB$ и~$CD$ пересекаются в~точке $E_1$;
продолжения сторон $BC$ и~$DE$~--- в~точке $A_1$.
Касательная, проведённая в~точке~$B$ к~описанной окружности треугольника
$B E_1 C$, пересекает $\omega$ в~точке~$B_1$;
аналогично определяется точка $D_1$.
Докажите, что  $B_1 D_1 \parallel AE$.
\end{problems}

