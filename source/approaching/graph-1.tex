% $date: 2014-11-26

\section*{Графы-1}

% $authors:
% - Юлий Тихонов

% $build$matter[print]: [[.], [.]]

\definition
\emph{Граф}~--- это конечный набор \emph{вершин}, и~некоторый набор
\emph{ребер}~--- неупорядоченных пар различных вершин.
\emph{Степень вершины}~--- количество ребер, в~которых она содержится.

\definition
\emph{Пути} и~\emph{циклы}~--- это цепочки из~вершин, последовательно
соединенных ребрами.
Циклы к~тому~же замкнуты.

\definition
\emph{Простые} пути и~циклы~--- проходят через каждую вершину не~более чем
по~одному разу (не~самопересекаются).

\begin{problems}

\item
\sp
Докажите, что если две вершины соединены путем, то~они соединены и~простым
путем.
\\
\sp
Пусть в~графе есть цикл.
Можно~ли утверждать, что в~нём есть простой цикл?
\\
\emph{Пункты сдаются одновременно.}

\end{problems}

\definition
\emph{Связный} граф~--- любые две вершины соединены путем.

\subsection*{Деревья}

\definition
\emph{Дерево}~--- связный граф, в~котором нет (простых) циклов.

\begin{problems}

\item
Пусть в~дереве $G$ есть хотя~бы две вершины.
\\
\sp\emph{Висячая вершина.}
Докажите, что в~$G$ есть хотя~бы одна вершина степени~1.
\\
\sp
Докажите, что количество ребер в~$G$ на~один меньше, чем количество вершин.
\\
\emph{Пункты сдаются одновременно.}

\item
\sp
Докажите, что в~$G$ есть хотя~бы две вершины степени 1.
\\
\sp
Докажите, что если в~графе есть (не~обязательно простой) цикл, проходящий
по~некоторому ребру ровно один раз, то~есть и~простой цикл.
\\
\sp
Докажите, что если в~графе есть (не~обязательно простой) цикл, проходящий
по~некоторому ребру нечетное раз, то~есть и~простой цикл.

\item\emph{Остовное дерево.}
Докажите, что в~любом связном графе можно удалить нес\-коль\-ко ребер так,
чтобы полученный граф стал деревом.

\end{problems}

\subsection*{Разное}

\begin{problems}

\itemx{$^\circ$}
Можно~ли на~плоскости нарисовать $15$ отрезков так, чтобы каждый пересекал
ровно 7 других?

\item
В~стране 101 город и~из~каждого города ведет не~менее 50 дорог.
Докажите, что из~любого города можно добраться в~любой другой
(возможно с~пересадками).

\item
В~лагере у~каждого пионера 20 друзей.
Как только пионер узнает новость, он~тут~же сообщает её~своим друзьям.
За~завтраком один из~пионеров узнал новость, и~к~обеду её~знал весь лагерь.
За~ужином двое пионеров поссорились.
На~следующий день за~завтраком пионеру (не~обязательно тому~же) сообщают
новость.
Докажите, что к~обеду о~ней опять узнает весь лагерь.

\item
В~стране любые два города соединены либо авиалинией, либо железной дорогой.
Министерство транспорта в~рамках программы экономии хочет закрыть один из~этих
видов перевозок.
\\
\sp
Докажите, что оно может это сделать так, чтобы из~любого города можно было
добраться в~любой другой.
\\
\sp
Каким числом пересадок можно гарантированно обойтись после этого?

\item
В~стране $2014$ городов, и~из~каждого выходит не~менее, чем $93$ дороги.
Известно, что из~каждого города можно добраться до~любого другого.
Докажите, что это можно сделать с~не~более, чем $63$ пересадками.

\item
На~какое наименьшее число частей нужно разрезать проволоку длиной
$12$~сантиметров, что~бы из~неё можно было сложить каркас кубика со~стороной
$1$~сантиметр?

\itemx{*}
В~королевстве 16 городов.
Король хочет соединить их~дорогами так, что~бы из~каждого города выходило
не~более 4 дорог и~из~любого можно было добраться до~любого другого, сделав
не~более 1 пересадки.
Сможет~ли он~это сделать?

\end{problems}

