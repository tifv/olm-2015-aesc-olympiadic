% $date: 2014-12-12

\section*{Подготовка к региону. Комбинаторика с числами}

% $authors:
% - Виктор Трещёв

% $build$matter[print]: [[.], [.]]

\begin{problems}

\item
На~доску выписаны 2011 чисел.
Оказалось, что сумма любых трех выписанных чисел также является выписанным
числом.
Какое наименьшее количество нулей может быть среди этих чисел?

\item
Даны 2011 ненулевых целых чисел.
Известно, что сумма любого из~них с~произведением оставшихся 2010 чисел
отрицательна.
Докажите, что если произвольным образом разбить все данные числа на~две группы
и~перемножить числа в~группах, то~сумма двух полученных произведений также
будет отрицательной.

\item
Прямую палку длиной 2~метра распилили на~$N$~палочек, длина каждой из~которых
выражается целым числом сантиметров.
При каком наименьшем $N$ можно гарантировать, что, использовав все получившиеся
палочки, можно, не~ломая их, сложить контур некоторого прямоугольника?

\item
На~окружности длины 2013 отмечены 2013 точек, делящих её~на~равные дуги.
В~каждой отмеченной точке стоит фишка.
Назовем \emph{расстоянием} между двумя точками длину меньшей дуги между этими
точками.
При каком наибольшем $n$ можно переставить фишки так, чтобы снова в~каждой
отмеченной точке было по~фишке, а~расстояние между любыми двумя фишками,
изначально удаленными не~более, чем на~$n$, увеличилось?

\end{problems}

