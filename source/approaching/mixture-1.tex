% $date: 2014-10-22

\section*{Разнобой-1}

% $authors:
% - Юлий Тихонов

% $build$matter[print]: [[.], [.], [.], [.]]
% $build$style[print]:
% - .[tiled4,-print]

\begin{problems}

\item
Можно~ли доску размером
\quad
\sp $10 \times 10$
\quad
\sp $12 \times 12$
\quad
клеток разрезать на~фигурки
\jeolmfigure[height=1.5ex]{t-tetramino}
из~четырех клеток?
Фигурки разрешается поворачивать и~переворачивать.
\emph{Пункты сдаются одновременно.}

\item
Двое играют на~доске
\quad
\sp $20 \times 14$
\quad
\sp $25 \times 62014$
\quad
клеток.
Каждый по~очереди отмечает квадрат по~линиям сетки (любого возможного размера)
и~закрашивает его.
Выигрывает тот, кто закрасит последнюю клетку.
Дважды закрашивать клетки нельзя.

\item
Марья-искусница умеет делать с~куском ткани следующую операцию: разрезать кусок
по~прямой линии на~две части, перевернуть одну из~частей на~другую сторону
и~сшить два куска в~один по~бывшему разрезу.
У~неё есть квадратный шелковый платок.
Может~ли Марья-искусница получить из~него правильный треугольник?

\item
Докажите, что если для положительных чисел $a$, $b$, $c$ выполнено
$a b + b c + c a \geq 12$, то~$a + b + c \geq 6$.

\item
Из~точки~$M$, двигающейся по~окружности, опускаются перпендикуляры $MP$ и~$MQ$
на~фиксированные диаметры $AB$ и~$CD$.
Докажите, что длина отрезка $PQ$ не~зависит от~положения точки~$M$.

\end{problems}

