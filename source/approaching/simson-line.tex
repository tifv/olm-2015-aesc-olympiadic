% $date: 2015-02-18

\section*{Прямая Симсона}

% $authors:
% - Юлий Тихонов

% $build$matter[print]: [[.], [.], [.], [.]]
% $build$style[print]:
% - .[tiled4,-print]

% $matter[-contained,no-header]:
% - verbatim: \setproblem{5}
% - .[contained]

% $matter[contained,-no-header]:
% - .[no-header]

Пусть $ABC$~--- треугольник, $\omega$~--- его описанная окружность, $P$~---
какая-то точка.
Проекции точки $P$ на стороны треугольника обозначим соответственно
$P_A$, $P_B$, $P_C$.

\begin{problems}

\item
Если $P$ лежит на $\omega$, то $P_A$, $P_B$, $P_C$ лежат на одной прямой.

\item
Если $P_A$, $P_B$, $P_C$ лежат на одной прямой, то $P$ лежит на $\omega$.

\end{problems}

Ещё задачи.

\begin{problems}

\item
Точку $P$, лежащую на описанной окружности треугольника $ABC$, отразили
относительно каждой из трёх сторон треугольника.
Докажите, что полученные три точки лежат на одной прямой.

\item
Точки $A$, $B$ и $C$ лежат на~одной прямой, точка $P$~--- вне этой прямой.
Докажите, что центры описанных окружностей треугольников $ABP$, $BCP$, $ACP$
и~точка~$P$ лежат на~одной окружности.

\item
Точка~$P$ движется по~описанной окружности треугольника $ABC$.
Докажите, что при этом прямая Симсона точки~$P$ относительно $ABC$
поворачивается на~угол, равный половине угловой величины дуги, пройденной $P$.

\item
Окружность с~центром в~точке~$I$, вписанная в~треугольник $ABC$, касается
сторон $AB$ и~$BC$ в~точках $C_1$ и~$A_1$ соответственно.
Окружность, проходящая через точки $B$ и~$I$, пересекает стороны $AB$ и~$BC$
в~точках $M$ и~$N$.
Докажите, что середина отрезка~$MN$ лежит на~прямой~$A_1 C_1$.

\item
В~треугольнике $ABC$ проведена биссектриса~$AD$ и из точки~$D$ опущены
перпендикуляры $DB'$ и $DC'$ на~прямые $AC$ и $AB$; точка~$M$ лежит
на~прямой~$B'C'$, причем $DM \perp BC$.
Докажите, что точка $M$ лежит на~медиане $A A_1$.

\end{problems}

