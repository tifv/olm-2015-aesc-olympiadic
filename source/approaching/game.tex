% $date: 2014-09-19

\section*{Математические игры}

% $authors:
% - Виктор Трещёв

% $build$matter[print]: [[.], [.]]

\begin{problems}

\item
Двое играют в~двойные шахматы: все фигуры ходят как обычно, но~каждый делает
по~два шахматных хода подряд.
Докажите, что первый может как минимум сделать ничью.

\item
В~центре квадрата сидит волк, а~в~каждой из~вершин~--- по~одной собаке.
Волк может бегать по~внутренности квадрата с~максимальной скоростью $v$,
а~собаки~--- только по~сторонам квадрата с~максимальной скоростью $1.5v$.
Известно, что волк задирает собаку, а~две собаки задирают волка.
Всегда~ли волк сможет выбежать из~квадрата?

\item
Ферзь стоит на~поле \texttt{c1}.
Двое игроков по~очереди передвигают его на~любое число поле вправо, вверх или
по~диагонали <<вправо-вверх>>.
Выигрывает тот, кто поставит ферзя на~поле \texttt{h8}.
Кто выиграет при правильной игре?

\item
Вершины правильного $2n$-угольника закрашены черной и~белой краской через одну.
Двое играют в~следующую игру.
Каждый по~очереди проводит отрезок, соединяющий вершины одинакового цвета.
Эти отрезки не~должны иметь общих точек (даже концов) с~проведенными ранее.
Побеждает тот, кто сделал последний ход.
Кто выигрывает при правильной игре: начинающий игру или его партнер?

\item
Дана клетчатая доска $10 \times 10$.
Два игрока играют в игру.
За~ход разрешается покрыть любые две соседние клетки доминошкой
(прямоугольником размером $1 \times 2$) так, чтобы доминошки не~перекрывались.
Проигрывает тот, кто не~может сделать ход.

\item
Играют двое, ходят по~очереди.
Первый ставит на~плоскости красную точку, второй в~ответ ставит на~свободные
места 10 синих точек.
Затем опять первый ставит на~свободное место красную точку, второй ставит
на~свободные места 10 синих, и~т.~д.
Первый считается выигравшим, если какие-то три красные точки образуют
правильный треугольник.
Может~ли второй ему помешать?

\item
В~углу шахматной доски стоит ладья.
Двое играют в~такую игру.
За~ход разрешается сходить ладьей по~шахматным правилам.
При этом ладья не~может ходить на~клетки, или <<пролетать>> над клетками,
в~которых она уже побывала или над которыми <<пролетала>>.
Проигрывает тот, кто не~может сделать ход.
Кто выигрывает при правильной игре?

\item
По~$n$ коробкам разложены $2n$ конфет.
Девочка и~мальчик по~очереди берут по~одной конфете, первой выбирает девочка.
Докажите, что мальчик может выбирать конфеты так, чтобы две последние конфеты
оказались из~одной коробки.

\end{problems}

