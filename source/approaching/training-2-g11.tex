% $date: 2014-12-05

\section*{Тренировочная олимпиада --- 2, 11 класс}

% $build$matter[print]: [[.], [.]]

\begin{problems}

\item
Натуральное число~$a$ имеет ровно четыре различных натуральных делителя
(включая 1 и~$a$).
Натуральное число~$b$ имеет ровно шесть различных натуральных делителей
(включая 1 и~$b$).
Может~ли число $c = a b$ иметь ровно пятнадцать различных натуральных делителей
(включая 1 и~$c$)?

\item
На~экране компьютера число 12.
Каждую секунду число на~экране умножают или делят либо на~2, либо на~3.
Результат действия возникает на~экране вместо записанного числа.
Ровно через минуту на~экране появилось число.
Могло~ли это быть число 54?

\item
Точка~$X$ расположена на~диаметре~$AB$ окружности радиуса~$R$.
Точки $K$ и~$N$ лежат на~окружности в~одной полуплоскости относительно $AB$,
а~$\angle KXA = \angle NXB = 60^\circ$.
Найдите длину отрезка~$KN$.

\item
Известно, что $A$~--- наибольшее из~чисел, являющихся произведением нескольких
натуральных чисел, сумма которых равна $2011$.
На~какую наибольшую степень тройки делится число~$A$?

\end{problems}

\begin{center}\small\sffamily
Окружной тур в~это воскресенье, 7 декабря.
Не~продолбайте!
\end{center}

