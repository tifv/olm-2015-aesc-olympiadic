% $date: 2014-12-03

\section*{Тренировочная олимпиада --- 1, 10 класс}

% $build$matter[print]: [[.], [.]]

\begin{problems}

\item
В~клетках таблицы $9 \times 9$ расставили все натуральные числа от~$1$ до~$81$.
Вычислили произведения чисел в~каждой строке таблицы и~получили набор из~девяти
чисел.
Затем вычислили произведения чисел в~каждом столбце таблицы и~также получили
набор из~девяти чисел.
Могли~ли полученные наборы оказаться одинаковыми?
% 2014, 10.6

\item
В~произвольный треугольник вписана окружность.
Проведем три касательные к~ней параллельно сторонам треугольника.
Докажите, что периметр образовавшегося шестиугольника не~превосходит $2/3$
периметра исходного треугольника.
% 2009, 10.5

\item
Докажите, что уравнение $l^2 + m^2 = n^2 + 3$ имеет бесконечно много решений
в~натуральных числах.
% 2012, 10.6

\item
Дана равнобокая трапеция $ABCD$ ($AD \parallel BC$).
На~дуге $AD$ (не~содержащей точек $B$ и~$C$) описанной окружности этой трапеции
произвольно выбрана точка $M$.
Докажите, что основания перпендикуляров, опущенных из~вершин $A$ и~$D$
на~отрезки $BM$ и~$CM$, лежат на~одной окружности.
% 2013, 10.5

\end{problems}

\begin{center}\small\sffamily
Окружной тур в~это воскресенье, 7 декабря.
Не~продолбайте!
\end{center}

