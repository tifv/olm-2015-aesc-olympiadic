% $date: 2014-11-28

\section*{Функция Эйлера}

% $authors:
% - Виктор Трещёв

% $build$matter[print]: [[.], [.]]

\definition
\emph{Функция Эйлера} $\phi(n)$ определяется как количество чисел от~$1$
до~$n$, взаимно простых с~$n$.

Также считается $\phi(1) = 1$.

\begin{problems}

\item
Найдите $\phi(p^\alpha)$, где $p$~--- простое, а $\alpha$~--- натуральное.

\item
Докажите \emph{мультипликативность} функции Эйлера: если $m$ и~$n$~--- взаимно
простые числа, то
\[
    \phi(mn) = \phi(m) \cdot \phi(n
.\]

\item
Пусть $n = p_1^{\alpha_1} p_2^{\alpha_2} \ldots p_k^{\alpha_k}$.
Докажите равенство
\[
    \phi (n)
=
    n
    \left(1 - \frac{1}{p_1}\right)
    \left(1 - \frac{1}{p_2}\right)
    \ldots
    \left(1 - \frac{1}{p_k}\right)
.\]
\emph{(Одно из решений использует формулу включений-исключений.)}
 
\item
Решите уравнения
\[
    \phi (x) = 2
;\qquad
    \phi (x) = 8
;\qquad
    \phi (x) = 12
;\qquad
    \phi (x) = 14
.\]

\item
Известно, что $(m, n) > 1$.
Что больше~--- $ \phi (m \cdot n)$ или $\phi (m) \cdot \phi(n)$?

\item
Докажите \textbf{Тождество Гаусса:}
\[
    \sum_{d \mid n} \phi(d) = n.
\]
Суммирование здесь ведется по всем натуральным $d$ таким, что $d \mid n$,
то~есть по~всем делителям числа~$n$.

\item\textbf{Теорема Эйлера.}
Если $(a, m) = 1$, то
\[
    a^{\phi(m)} \equiv 1 \pmod m
.\]

\item
Cуществует~ли степень тройки, заканчивающаяся на~$0001$?

\item
Даны числа $a$, $b$, $m$, причем $(a, m) = 1$.
Найдите корень сравнения $a x + b \equiv 0 \pmod m$.

\item
Докажите, что при любом нечетном $n$ число $2^{n!} - 1$ делится на~$n$.

\item
Докажите, что для любого числа $n$ найдется число, делящееся на~$n$, сумма цифр
которого равна $n$.

\item
Найдите сумму всех правильных несократимых дробей со~знаменателем~$n$.

\end{problems}

