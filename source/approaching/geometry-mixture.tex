% $date: 2014-12-17

\section*{Планиметрический разнобой}

% $authors:
% - Юлий Тихонов

% $build$matter[print]: [[.], [.]]
% $build$matter[print,full-version]: [[.], [.]]

% $matter[full-version,-no-header]:
% - .[header,multidate]
% $matter[full-version]:
% - verbatim: \let\ifradicalaxisexplained\iffalse
% - .[-full-version]
% - ../geometry-mixture-more[contained]

\providecommand{\ifradicalaxisexplained}{\iftrue}


\subsection*{Прямая Эйлера}

\begin{problems}

\item
Пусть $O$~--- центр описанной окружности треугольника $ABC$,
$H$~--- его ортоцентр,
$A_1$~--- середина стороны~$BC$.
Докажите, что $AH = 2 O A_1$.

\item
Докажите, что в~треугольнике $ABC$ точка пересечения высот $H$,
точка пересечения медиан $M$ и~центр описанной окружности $O$ лежат на~одной
прямой, причем $HM = 2 MO$.

\item
Дан вписанный четырехугольник $ABCD$.
$H_A$ и~$H_B$~--- ортоцентры треугольников $BCD$ и~$ACD$ соответственно.
Докажите, что $H_A H_B = AB$.

\end{problems}


\subsection*{Радикальные оси}

\ifradicalaxisexplained

\definition
Величина $OA^2 - r^2$ называется \emph{степенью точки $A$} относительно
окружности с~радиусом $r$ и центром~$O$.

\begin{problems}

\item
\spx{$^\circ$}
Докажите, что геометрическое место точек с равными степенями относительно двух
неконцентрических окружностей есть прямая, перпендикулярная их линии центров.
Эта прямая называется \emph{радикальной осью} двух окружностей.
\\
\spx{$^\circ$}
Докажите, что радикальные оси трех окружностей, центры которых не лежат на
одной прямой, пересекаются в одной точке.
Эта точка называется \emph{радикальным центром} трёх окружностей.

\else % \ifradicalaxisexplained

\stepcounter{jeolmproblem}

\begin{problems}

\fi % \ifradicalaxisexplained

\item
\sp
Докажите, что радикальная ось делит отрезок общей касательной двух окружностей
пополам.
\\
\sp
В угол вписаны две окружности.
Одна из них касается сторон угла в точках $A$ и $B$, а другая~--- в точках $C$
и $D$ соответственно.
Докажите, что прямая $AD$ высекает на этих окружностях равные хорды.

\item
$AB$~--- диаметр окружности~$\omega$, $C$~--- точка на~ней~же.
Окружность с~центром в~точке~$C$ касается прямой~$AB$ в~точке~$D$ и~пересекает
$\omega$ в~точках $E$, $F$.
Докажите, что отрезок~$EF$ точкой пересечения делит отрезок~$CD$ пополам.

\item
Внутри выпуклого многоугольника расположено несколько попарно непересекающихся
кругов различных радиусов.
Докажите, что многоугольник можно разрезать на маленькие многоугольники так,
чтобы все они были выпуклыми и в каждом из них содержался ровно один из данных
кругов.

\end{problems}


\subsection*{Разнобой}

\begin{problems}

\item
Через вершину $B$ остроугольного треугольника $ABC$ проведено две окружности,
которые касаются стороны $AC$ в точках $A$ и $C$ и пересекаются вторично
в~точке~$M$.
\\
\sp
Докажите, что $M$ лежит на медиане треугольника, выходящей из вершины $B$.
\\
\sp
Докажите, что $A$, $C$, $M$ и ортоцентр треугольника $H$ лежат на одной
окружности.

\item
В параллелограме $ABCD$ $O$~--- точка пересечения диагоналей.
Окружность, проходящая через точки $A$, $O$ и $B$ касается стороны $BC$.
Докажите, что описанная окружность $\triangle BOC$ касается $CD$.

\item
В правильном треугольнике $ACB$ на стороне $AC$ взяли такие $n$ различных точек
$P_i$, $i = 1, \ldots, n$ так, что $A P_1 = P_1 P_2 = \ldots = P_{n-1} P_{n} = P_{n} C$.
На стороне $BC$ выбрана такая точка $K$, что $P_1 K \parallel B A$.
Покажите, что
\(
    \sum_{i = 1}^{n}
        \angle B P_i K
=
    30^{\circ}
\).

\item
\spx{$^\circ$}
Докажите, что точка, симметричная ортоцентру $H$ треугольника $ABC$
относительно середины стороны лежит на описанной окружности треугольника $ABC$,
и диаметрально противоположна вершине треугольника.
\\
\sp
Докажите, что $A$, $C$, $H$ и проекция $H$ на медиану треугольника, выходящую
из вершины $B$, лежат на одной окружности.

\item\emph{Лемма Архимеда.}
Окружность $s_1$ касается окружности $s$ внутренним образом в точке $N$.
Хорда $AB$ окружности $s$ касается окружности $s_1$ в точке $M$.
Докажите, что $MN$ делит дугу $AB$, не содержащую точку $N$, пополам.

\item
Хорды $AC$ и $BD$ окружности с центром $O$ пересекаются в точке $K$.
Пусть $M$ и $N$~--- центры окружностей, описанных около треугольников $AKB$ и
$CKD$ соответственно.
Докажите, что $OM = KN$.

\end{problems}

