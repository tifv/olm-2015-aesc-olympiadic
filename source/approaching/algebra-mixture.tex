% $date: 2015-01-21

\section*{Алгебраический разнобой}

% $authors:
% - Юлий Тихонов

% $build$matter[print]: [[.], [.], [.], [.]]
% $build$style[print]:
% - .[tiled4,-print]

% $build$matter[print,full-version]: [[.], [.]]
% $build$style[print,full-version]:
% - .

% $matter[full-version,-no-header]:
% - .[header,multidate]
% $matter[full-version]:
% - .[-full-version]
% - ../algebra-mixture-more[contained]

\subsection*{Простые задачи}

\begin{problems}

\item
На~плоскости выбраны пять различных точек с~целыми координатами.
Докажите, что можно выбрать две из~них так, чтобы середина отрезка между ними
также имела целые координаты.

\item
Петя выбрал два натуральных числа, возвел их~в~квадрат и~сложил.
Последние две цифры его результата~--- $27$.
Докажите, что он~ошибся.

\item
Квадратный трехчлен $f(x)$ таков, что уравнение $(f(x))^2 - 4 = 0$ имеет
хотя~бы $3$ корня.
Докажите, что уравнение $f(x) = 0$ имеет два корня.

\item
Докажите, что
\(
    \dfrac{1}{1 \cdot 2}
    +
    \dfrac{1}{2 \cdot 3}
    + \ldots +
    \dfrac{1}{99 \cdot 100}
<
    1
\).

\item
Докажите, что из~любой бесконечной арифметической прогрессии натуральных чисел
можно убрать некоторые члены так, чтобы осталась бесконечная геометрическая
прогрессия.

\end{problems}


\subsection*{Просто задачи}

\begin{problems}

\item
Найдите все $x$, при которых уравнение $x^2 + y^2 + z^2 + 2 x y z = 1$
имеет решение относительно $z$ при любом вещественном $y$.
\emph{\small Округ, 2003, 10.5}

\item
Натуральное число $b$ назовём \emph{удачным}, если для любого натурального $a$
такого, что $a^5$ делится на~$b^2$, число $a^2$ делится на~$b$.
Найдите количество удачных натуральных чисел, меньших 2010.
\emph{\small Регион, 2010, 10.4}

\item
Положительные числа $x_1$, $x_2$, $\ldots$, $x_{2009}$ удовлетворяют равенствам
\(
    x_1^2 - x_1 x_2 + x_2^2
=
    x_2^2 - x_2 x_3 + x_3^2
= \ldots =
    x_{2008}^2 - x_{2008} x_{2009} + x_{2009}^2
=
    x_{2009}^2 - x_{2009} x_1 + x_1^2
\).
Докажите, что числа $x_1, x_2, \ldots, x_{2009}$ равны.
\emph{\small Регион, 2009, 10.7}

\item
Ненулевые числа $a$, $b$, $c$ таковы что любые два из~трёх уравнений
$a x^{11} + b x^4 + c = 0$, $b x^{11} + c x^4 + a = 0$,
$c x^{11} + a x^4 + b = 0$ имеют общий корень.
Докажите, что все три уравнения имеют общий корень.
\emph{\small Регион, 2011, 10.4}

\item
Три натуральных числа таковы, что произведение любых двух из~них делится
на~сумму этих двух чисел.
Докажите, что эти три числа имеют общий делитель, больший единицы.
\emph{\small Округ, 2004, 9.4}

\end{problems}

