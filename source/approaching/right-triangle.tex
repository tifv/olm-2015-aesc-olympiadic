% $date: 2014-10-15

\section*{Прямоугольный треугольник}

% $authors:
% - Фёдор Бахарев

% $build$matter[print]: [[.], [.]]
% $build$matter[tiled4]: [[.], [.], [.], [.]]

В~треугольнике $ABC$ с~прямым углом $\angle C$ построены:
$CH$~--- высота,
$O$, $O_1$, $O_2$~--- центры окружностей, вписанных в~треугольники
$ABC$, $ACH$ и $CBH$ соответственно,
$r$, $r_1$, $r_2$ --- их~радиусы.
Прямая~$O_1 O_2$ пересекает стороны $AC$ и~$BC$ в~точках $U$ и~$V$
соответственно.
Прямые $C O_1$ и $C O_2$ пересекают сторону~$AB$ в~точках $P$ и~$Q$
соответственно.
Докажите следующие утверждения:

\subsection*{Уголки}

\begin{problems}

\item
Треугольники $ACQ$ и $BCP$ равнобедренные.

\item
\sp
Точка~$O$~--- ортоцентр (точка пересечения высот) треугольника $C O_1 O_2$.
\\
\sp
А еще $CU = CV$.

\item
Точки $A$, $O_1$, $O_2$, $B$ лежат на~одной окружности.

\item
Точки $A$, $P$, $O$, $C$, чудесным образом, тоже лежат на~одной окружности.

\item
Описанные окружности треугольников $A C O_1$ и~$B C O_2$ касаются в~точке~$C$
с~общей касательной $OC$.

\item
Неожиданно $P O_2 \parallel A O$ и $Q O_1 \parallel B O$.

\end{problems}


\subsection*{Подобие}

\begin{problems}

\item
Треугольники $ACH$ и~$BCH$ подобны $ABC$ с~коэффициентами $k_1$ и~$k_2$ такими,
что $k_1^2 + k_2^2 = 1$.

\item
Для радиусов вписанных окружностей выполнено равенство $r^2 = r_1^2 + r_2^2$.

\item
Треугольники $O_1 O_2 H$ и~$ABC$ подобны.
\qquad
\problem
Внезапно, $O_1 O_2 = C O$.

\end{problems}

\statement
Площадь $S$ любого треугольника выражается через радиус $r$ его вписанной
окружности и его стороны $a$, $b$ и $c$:
\[
    S = \frac{r \cdot (a + b + c)}{2}
\]

\begin{problems}

\item
И~еще одно волшебное наблюдение: $r + r_1 + r_2 = CH$.

\end{problems}

