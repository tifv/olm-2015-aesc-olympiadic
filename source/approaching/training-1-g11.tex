% $date: 2014-12-03

\section*{Тренировочная олимпиада --- 1, 11 класс}

% $build$matter[print]: [[.], [.]]

\begin{problems}

\item
Последовательность чисел строится следующим образом.
Первое число в~ней равно 2.
Каждое последующее число равно сумме кубов цифр предыдущего числа.
Вася утверждает, что среди чисел этой последовательности встретятся два
одинаковых числа, Коля~--- что этого никогда не~произойдет.
Кто из~них прав?
% 2009, 11.4

\item
Докажите, что для всех $x$ выполняется неравенство:
\[
    x^2 + x \sin x + x^2 \cos x + 0.5 > 0
\,.\]

\item
В~правильной четырехугольной усеченной пирамиде середина $N$ ребра $B_1 C_1$
верхней грани $A_1 B_1 C_1 D_1$ соединена с~серединой $M$ ребра $AB$ нижней
грани $ABCD$ (буквы в~гранях соответствуют).
Докажите, что проекции ребер $B_1 C_1$ и~$AB$ на~прямую $MN$ равны между собой.

\item
Две окружности касаются внешним образом.
$A$~--- точка касания их~общей внешней касательной с~одной из~окружностей,
$B$~--- точка той~же окружности, диаметрально противоположная точке $A$.
Докажите, что длина касательной, проведенной из~точки $B$ ко второй окружности,
равна диаметру первой окружности.

\end{problems}

\begin{center}\small\sffamily
Окружной тур в~это воскресенье, 7 декабря.
Не~продолбайте!
\end{center}

