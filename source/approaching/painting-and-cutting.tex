% $date: 2014-09-12

\section*{Раскраски и разрезания}

% $authors:
% - Виктор Трещёв

% $build$matter[print]: [[.], [.]]

\begin{problems}

\item
Квадрат $4 \times 4$ разделен на~16 клеток.
Раскрасьте эти клетки в~черный и~белый цвета так, чтобы у~каждой черной клетки
было три белых соседа, а~у~каждой белой клетки был ровно один черный сосед.
(Соседними считаются клетки, имеющие общую сторону.)

\item
Клетки доски $10 \times 10$ раскрашены в~красный, синий и~белый цвета.
Каждые две клетки с~общей стороной раскрашены в~разные цвета.
Известно, что красных клеток 20.
\\
\sp Докажите, что из доски всегда можно вырезать 30 прямоугольников, каждый
из~которых состоит из~двух клеток~--- белой и~синей.
\\
\sp Приведите пример раскраски, когда можно вырезать 40 таких прямоугольников.
\\
\sp Приведите пример раскраски, когда нельзя вырезать больше 30 таких
прямоугольников.

\item
При каких $n$ можно раскрасить в~три цвета все ребра $n$-угольной призмы
(основания призмы~--- $n$-угольники) так, что в~каждой вершине сходятся все
три цвета и~у~каждой грани (включая основания) есть стороны всех трех цветов?

\item
Бесконечная клетчатая доска раскрашена в~2014 цветов
(каждая клеточка~--- в~один из~цветов).
Докажите, что найдутся четыре клеточки одного цвета, расположенные в~вершинах
прямоугольника со~сторонами, параллельными сторонам клеточек.

\item
Назовем \emph{крокодилом} шахматную фигуру, ход которой заключается в~прыжке
на~$m$ клеток по~вертикали или по~горизонтали, и~потом на~$n$ клеток
в~перпендикулярном направлении.
Докажите что для любых $m$ и~$n$ можно так раскрасить бесконечную клетчатую
доску в~два цвета (для каждых конкретных $m$ и~$n$ своя раскраска), что всегда
две клетки, соединенные одним ходом крокодила, будут покрашены в~разные цвета.

\item
Дана бесконечная клетчатая бумага со~стороной клетки, равной единице.
Расстоянием между двумя клетками называется длина кратчайшего пути ладьи
от~одной клетки до~другой (считается путь центра ладьи).
В~какое наименьшее число красок нужно раскрасить доску (каждая клетка
закрашивается одной краской), чтобы две клетки, находящиеся на~расстоянии 6,
были всегда окрашены разными красками?

\item
Из~квадратной клетчатой доски $n \times n$, где $n$~--- нечетное, вырезали
одну угловую клетку.
Можно~ли полученную фигуру полностью уложить фигурками домино так, чтобы
в~укладке было поровну горизонтально и~вертикально расположенных домино?

\end{problems}

