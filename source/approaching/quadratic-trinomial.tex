% $date: 2015-02-25

\section*{Квадратный трёхчлен}

% $build$matter[print]: [[.], [.], [.], [.]]
% $build$style[print]:
% - .[tiled4,-print]

\begin{problems}

\item
Один из двух приведенных квадратных трехчленов имеет два корня меньших тысячи,
другой~--- два корня больших тысячи.
Может ли сумма этих трехчленов иметь один корень меньший тысячи,
а другой~--- больший тысячи?

\item
Даны вещественные числа $x_1$, $x_2$, $y_1$, $y_2$ такие, что
$x_1 + x_2 = y_1 + y_2$ и $x_1 x_2 = y_1 y_2$.
Докажите, что наборы $\{x_1, x_2\}$ и $\{y_1, y_2\}$ совпадают.

\item
На плоскости по трем прямым с постоянными скоростями движутся три точки
(каждая по своей прямой).
Докажите, что или эти точки всегда лежат на одной прямой или
найдется не более двух моментов,
когда через них можно провести прямую.

\item
Пусть $x_1$ и $x_2$~--- корни квадратного уравнения $x^2 - 3 x - 5 = 0$.
Составьте квадратное уравнение, корнями которого являются числа:
\\
\sp $x_1 + \frac{1}{x_1}$ и $x_2 + \frac{1}{x_2}$;
\qquad
\sp $x_1 + \frac{1}{x_2}$ и~$x_2 + \frac{1}{x_1}$.

\item
Известно, что $(a + b + c) \cdot c < 0$.
Докажите, что $b^2 > 4 a c$.

\item
В квадратном уравнении $x^2 + p x + q = 0$ коэффициенты $p$ и $q$ независимо
пробегают все значения от $-1$ до $1$.
Найдите множество значений, которые могут при этом принимать действительные
корни этого уравнения.

\item
Рассмотрим графики функций $y = x^2 + p x + q$, которые пересекают оси
координат в трех различных точках.
Докажите, что все окружности, описанные около треугольников с вершинами в этих
точках, имеют общую точку.

\item
Может ли у квадратного трехчлена с целыми коэффициентами дискриминант равняться
$2007$?

\item
Прожектор установлен так, что освещает внутренность параболы (ось параболы
не обязательно параллельна $Ox$).
Можно ли конечным числом таких прожекторов осветить всю плоскость?

\item
Есть две параболы $y = x^2 + x - 41$ и $x = y^2 + y - 40$.
Докажите, что точки их пересечения лежат на одной окружности.

\end{problems}

