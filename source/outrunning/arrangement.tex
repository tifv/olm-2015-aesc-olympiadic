% $date: 2014-09-10

\section*{Упорядочивание}

% $build$matter[print]: [[.], [.]]

\begin{problems}

\item
Докажите, что цифры любого шестизначного числа можно переставить так, что сумма
первых трех будет отличаться от~суммы остальных не~более, чем на~$9$.

\item
Пусть каждое из~$2n$ различных натуральных чисел $a_1, \ldots, a_{2n}$
не~превосходит $n^2$ ($n > 2$).
Докажите, что среди попарных разностей найдутся хотя~бы три равные.

\item
Каждое из~семи различных натуральных чисел не~превосходит $1706$.
Докажите, что среди них найдутся три, $a$, $b$ и~$c$, такие, что $a < b + c < 4a$.

\item
В~таблице $10 \times 10$ записаны числа от~$1$ до~$100$.
В~каждой строке выбирается третье по~величине число.
Докажите, что сумма этих чисел не~меньше суммы чисел хотя~бы в~одной из~строк.

\item
Обозначим через $a$ и~$A$ соответственно наименьшее и~наибольшее из~$n$
различных натуральных чисел.
Докажите, что их~НОК не~меньше $na$ и~их~НОД не~больше $A / n$.

\item
Докажите, что из~$69$ различных натуральных чисел, не~превосходящих $100$,
можно выбрать четыре $a, b, c, d$ так, что $a < b < c$ и~$a + b + c = d$.
Верно~ли это для $68$ чисел?

\item
Докажите, что из~$25$ различных натуральных чисел можно выбрать два, сумма
и~разность которых не~совпадают ни~с~одним из~оставшихся $23$.

\end{problems}

