% $date: 2015-01-28

\section*{Агаханов пикчерз представляет}

\begin{problems}

\item
Известно, что $(a + b + c) \cdot c < 0$.
Докажите неравенство $b^2 > 4 a c$.

\item
Докажите, что стороны любого неравнобедренного треугольника можно либо все
увеличить, либо все уменьшить на~одну и~ту~же величину так, чтобы получился
прямоугольный треугольник.

\item
Длины сторон многоугольника равны $a_1, \ldots, a_n$.
Квадратный трехчлен $f(x)$ таков, что $f(a_1) = f(a_2 + \ldots + a_n)$.
Докажите, что если $A$~--- сумма длин нескольких сторон многоугольника, $B$~---
сумма длин остальных его сторон, то~$f(A) = f(B)$.

\item
Приведённый квадратный трехчлен с~целыми коэффициентами в~трех последовательных
целых точках принимает простые значения.
Докажите, что он~принимает простое значение по~крайней мере еще в~одной целой
точке.

\item
На~оси~$Ox$ произвольно расположены различные точки $x_1, \ldots, x_n$, $n
\geqslant 3$.
Построены все параболы, задаваемые приведёнными квадратными трехчленами
и~пересекающие ось~$Ox$ в~данных точках (и~не~пересекающие её~в~других точках).
Пусть $y = f_1(x)$, $\ldots$, $y = f_m(x)$~--- соответствующие параболы.
Докажите, что парабола $y = f_1(x) + \ldots + f_m(x)$ пересекает ось~$Ox$
в~двух точках.

\item
Для квадратного трехчлена $f(x)$ и~некоторых действительных чисел $a$, $b$, $c$
выполнены равенства: $f(a) = b + c$, $f(b) = a + c$, $f(c) = a + b$.
Докажите, что среди чисел $a$, $b$ и~$c$ есть равные.

\item
Квадратный трехчлен $f(x)$ таков, что уравнение $f(x) = x$ не~имеет
действительных корней.
Докажите, что $f(f(x)) = x$ также не~имеет действительных корней.

\item
Даны два квадратных трехчлена, имеющих корни.
Известно, что если в~них поменять местами коэффициенты при $x^2$, то~получатся
трехчлены, не~имеющие корней.
Докажите, что если в~исходных трехчленах поменять местами коэффициенты при $x$,
то~получатся трехчлены, имеющие корни.

\item
Пусть $f(x) = x^2 + x + p$, $p$~--- натуральное.
Докажите, что если $f(0)$, $f(1)$, \dots, $f\bigl( [p / 3] \bigr)$~---
простые числа, то~все числа $f(0)$, $f(1)$, \dots, $f(p-2)$ также простые.

\end{problems}

