% $date: 2014-10-29

\section*{Порядки}

\definition
Пусть $(a, n) = 1$.
Наименьшее натуральное число~$k$, для которого $a^k \equiv 1 \pmod n$, называют
\emph{порядком} числа~$a$ по~модулю~$n$.
Будем обозначать его через $\exp_n(a)$.

\begin{problems}

\item
\sp
Докажите, что для простого $p$, $\exp_p(a)$ является делителем
числа $p - 1$.
\\
\sp
Докажите, что $\exp_n(a)$ делит $\varphi(n)$.

\item
Докажите, что если простое число $p$ является делителем числа
$a^4 + a^3 + a^2 + a + 1$ для некоторого $a$, то~$p = 5$ или
$p \equiv 1 \pmod 5$.

\item
Докажите, что ни~при каком целом $a$ число $a^2 + a + 1$ не~кратно
\\
\sp $5$;
\quad
\sp $17$;
\quad
\sp $6m - 1$, где $m$~--- натуральное число.

\item
Пусть $p$ простое.
Докажите, что число $p^p - 1$ имеет простой делитель, сравнимый с~единицей
по~модулю~$p$.

\item
Докажите, что для всяких натуральных $a > 1$ и $n > 1$ число $\varphi(a^n - 1)$
делится на~$n$.

\item
\sp
Докажите, что любой нечетный простой делитель числа $a^2 + 1$ имеет вид
$p = 4k + 1$.
\sp
Докажите, что если $p$~--- нечетный простой делитель числа $a^{2^n} + 1$,
то~$p - 1$ делится на~$2^{n + 1}$.

\item
Найти все пары $(p, q)$ простых чисел такие, что число $2^p - 1$ делится
на~$q$, и~среди простых делителей числа $q - 1$ имеются только числа
$2$, $3$, $5$ и~$7$.

\item
Пусть $2^n + 1$ делится на~$n$.
Докажите, что $n$ делится на~$3$.

\item
Найдите все натуральные~$n$, для которых $2^n + 1$ кратно $n^2$.

\item
Докажите, что ни~для какого натурального $n > 2$ число $2^n - 1$ не~делится
на~$n$.

\end{problems}

