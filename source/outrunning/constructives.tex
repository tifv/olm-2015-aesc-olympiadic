% $date: 2015-02-27

\section*{Конструктивы и конструкции}

\begin{problems}

\item
Существует~ли в~пространстве замкнутая самопересекающаяся ломаная, которая
пересекает каждое свое звено ровно один раз, причем в~его середине?

\item
Имеется $120$ мешков по~сто монет в~каждом.
В~одном из~них лежат только фальшивые монеты, в~остальных только настоящие.
Фальшивая монета весит $9$ граммов, настоящая $10$ граммов.
Как найти, с~помощью трех взвешиваний на~весах (не~чашечных), мешок
с~фальшивыми монетами, если за~одно взвешивание на~весы можно класть не~более
$100$ монет?

\item
Можно~ли на~плоскости расположить бесконечное множество одинаковых кругов так,
чтобы любая прямая пересекала не~более двух кругов?

\item
Какое наибольшее количество подграфов $K_3$ (треугольников) можно выбрать
в~графе $K_{3\times n}$, так, чтобы никакие два подграфа не~пересекались
по~ребру ($K_{3\times n}$~--- граф, состоящий из~трех долей по~$n$ вершин
в~каждой, такой, что между любыми двумя вершинами из~разных долей ребро
проведено, из~одной доли~--- не~проведено)?

\item
Существуют~ли три попарно различных ненулевых целых числа, сумма которых равна
нулю, а~сумма тринадцатых степеней которых является квадратом некоторого
натурального числа?

\item
Загадано число от~$1$ до~$144$.
Разрешается выделить одно подмножество множества чисел от~$1$ до~$144$
и~спросить, принадлежит~ли ему загаданное число.
За~ответ <<да>> надо заплатить $2$ рубля, за~ответ <<нет>>~--- $1$ рубль.
Какая наименьшая сумма денег необходима для того, чтобы наверняка угадать
число?

\item
На~прямоугольном столе разложено несколько одинаковых квадратных листов бумаги
так, что их~стороны параллельны краям стола (листы могут перекрываться).
Докажите, что можно воткнуть несколько булавок таким образом, что каждый лист
будет прикреплен к~столу ровно одной булавкой.

%\item
%В~турнире по~теннису $n$ участников хотят провести парные (двое на~двое) матчи
%так, чтобы каждый из~участников имел своим противником каждого из~остальных
%ровно в~одном матче.
%При каких $n$ возможен такой турнир?

\item
В~$99$ ящиках лежат яблоки и~апельсины.
Докажите, что можно так выбрать $50$ ящиков, что в~них окажется не~менее
половины всех яблок и~не~менее половины всех апельсинов.

\end{problems}

