% $date: 2014-09-12

\section*{Упорядочивание. КомбиГеом}

% $authors:
% - Олег Орлов
% - Антон Гусев

% $build$matter[print]: [[.], [.]]

\begin{problems}

\item
На плоскости отмечено $2 n + 2$ точки, никакие три из которых не лежат
на одной прямой.
Докажите, что можно выбрать из них две так, что прямая, проходящая через них
делит остальные $2 n$ точек поровну.

\item
На плоскости расположено $4 n$ точек, никакие три из которых не лежат на одной
прямой.
Докажите, что можно выбрать $n$ непересекающихся четырехугольников
(не обязательно выпуклых) с вершинами в этих точках.

\item
На плоскости расположены $2 n + 3$ точки так, что никакие три не лежат
на одной прямой и никакие четыре не лежат на одной окружности.
Докажите, что среди них найдутся три, окружность проходящая через которые
содержит внутри ровно $n$ из этих точек.
    
\item
На прямой дано $2 n + 1$ отрезков.
Известно, что каждый пересекается не менее, чем с $n$ из оставшихся.
Доказать, что найдется отрезок, который пересекается со всеми.

\item
На плоскости дано $\left[\frac{4}{3}n\right]$ прямоугольников со сторонами
параллельными линиям сетки.
Причем известно, что каждый пересекается хотя бы с $n$ из оставшихся.
Докажите, что существует прямоугольник, который пересекается со всеми.

\end{problems}

