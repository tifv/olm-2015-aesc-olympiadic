% $date: 2015-02-04

\section*{Региональный разнобой}

\emph{Составлен из задач для 9 класса регионального этапа Всероссийской
олимпиады по математике~--- 2015.}

\begin{problems}

\item
За~круглым столом сидят 2015 человек, каждый из~них либо рыцарь, либо лжец.
Рыцари всегда говорят правду, лжецы всегда лгут.
Им~раздали по~одной карточке, на~каждой карточке написано по~числу;
при этом все числа на~карточках различны.
Посмотрев на~карточки соседей, каждый из~сидящих за~столом сказал:
<<Мое число больше, чем у~каждого из~двух моих соседей>>.
После этого $k$ из~сидящих сказали:
<<Мое число меньше, чем у~каждого из~двух моих соседей>>.
При каком наибольшем $k$ это могло случиться?

\item
Назовем натуральное число $k$ интересным, если сумма его цифр~--- простое
число.
Какое наибольшее количество интересных чисел может быть среди пяти подряд
идущих натуральных чисел?

\item
Правильный треугольник со~стороной~3 разбит на~девять правильных треугольных
клеток со~стороной~1.
В~этих клетках изначально записаны нули.
За~один ход можно выбрать два числа, находящиеся в~соседних по~стороне клетках,
и~либо прибавить к~обоим по~единице, либо вычесть из~обоих по~единице.
Петя хочет сделать несколько ходов так, чтобы после этого в~клетках оказались
записаны в~некотором порядке последовательные натуральные числа
$n, n + 1, \ldots, n + 8$.
При каких $n$ он~сможет это сделать?

\item
В~неравнобедренном треугольнике $ABC$ провели биссектрисы угла $ABC$ и~угла,
смежного с~ним.
Они пересекли прямую~$AC$ в~точках $B_1$ и~$B_2$ соответственно.
Из~точек $B_1$ и~$B_2$ провели касательные к~окружности, вписанной
в~треугольник $ABC$, отличные от~прямой~$AC$.
Они касаются этой окружности в~точках $K_1$ и~$K_2$ соответственно.
Докажите, что точки $B$, $K_1$ и~$K_2$ лежат на~одной прямой.

\item
Числа $a$, $b$, $c$ и~$d$ таковы, что $a^2 + b^2 + c^2 + d^2 = 4$.
Докажите, что $(2 + a) \cdot (2 + b) \geq c d$.

\item
Петя хочет выписать все возможные последовательности из~100 натуральных чисел,
в~каждой из~которых хотя~бы раз встречается тройка, а~любые два соседних члена
различаются не~больше, чем на~1.
Сколько последовательностей ему придется выписать?

\end{problems}

