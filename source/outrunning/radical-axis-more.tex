% $date: 2014-12-05

\section*{Степень точки. Радикальные оси. Добавка}

\begin{problems}

\item
Пусть вписанная окружность треугольника $ABC$ касается сторон $AB$, $AC$, $BC$
в~точках $C_1$, $B_1$, $A_1$.
Докажите, что средние линии треугольников $A_1 CB_1$ и~$A_1 B C_1$
соответственно параллельные сторонам $A_1 B_1$ и~$A_1 C_1$, а~также серединный
перпендикуляр к~$BC$ пересекаются в~одной точке.
\emph(А~что будет, если рассмотреть степень точки относительно окружности
нулевого радиуса?).

\item
На~плоскости даны точка~$X$ внутри окружности~$\omega$ и~точка~$A$ вне её.
Через точку~$X$ проводятся всевозможные хорды $BC$.
Найдите геометрическое место центров описанных окружностей треугольников $ABC$.
% извлекаем свойства

\item
Дан остроугольный треугольник $ABC$.
Точки $M$ и~$N$~--- середины сторон $AB$ и~$BC$ соответственно, точка~$H$~---
основание высоты, опущенной из~вершины~$B$.
Описанные окружности треугольников $AHN$ и~$CHM$ пересекаются в~точке~$P$
(отличной от~$H$).
Докажите, что прямая~$PH$ проходит через середину отрезка~$MN$.
% нетривиальный подсчёт степени точки

\item
Прямая~$OA$ касается окружности в~точке~$A$, а~хорда~$BC$ параллельна $OA$.
Прямые $OB$ и~$OC$ вторично пересекают окружность в~точках $K$ и~$L$.
Докажите, что прямая~$KL$ делит отрезок~$OA$ пополам.
% достраиваем окружность, которой нет в условии задачи.
% рассмотрим описанную окружность треугольника $OKL$

\item\emph{Окружность Конвея.}
На~прямых $AB$ и~$AC$ за~точку~$A$ отложили точки $A_1$ и~$A_2$ так, что
$A A_1 = A A_2 = BC$.
Аналогично построили точки $B_1$, $B_2$, $C_1$, $C_2$.
Докажите, что шесть точек $A_1$, $A_2$, $B_1$, $B_2$, $C_1$, $C_2$ лежат
на~одной окружности.

\item
Вписанная в~треугольник $ABC$ окружность~$\omega$ касается стороны~$BC$
в~точке~$A_1$.
$I_A$~--- центр вневписанной окружности, касающейся стороны~$BC$.
$M$~--- середина отрезка $A_1 I_A$.
Докажите, что длина касательной из~$M$ к~$\omega$ равна $MB$.

\item
$I$~--- точка пересечения биссектрис треугольника $ABC$.
Пусть $K$~--- точка пересечения перпендикуляра к~$BI$, проведенного в~точке
$I$, и~прямой~$AC$.
Докажите, что основание перпендикуляра, опущенного из~$I$ на~$BK$ лежит
на~описанной окружности треугольника $ABC$.
% достраиваем две окружности, которых нет в условии задачи.
% на $BI$ как на диаметре и описанную окружность треугольника $AIC$

\item 
Через точку на~медиане~$AM$ треугольника $ABC$ проведены окружности
$\omega_B$, $\omega_C$, касающиеся прямой~$BC$ в~точках $B$, $C$ и~пересекающие
стороны $AB$ и~$AC$ в~точках $X$ и~$Y$, всё соответственно.
Докажите, что описанная окружность треугольника $AXY$ касается $\omega_B$
и~$\omega_C$.
% извлекаем свойства
% в конце дорешивается одной инверсией/тривиальным счётом уголков
% можно без инверсии гомотетией
% можно просто углы посчитать

\end{problems}

