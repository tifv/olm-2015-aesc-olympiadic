% $date: 2014-11-26

\section*{Комбинаторная геометрия и Петер Густав Лежён Дирихле}

\begin{problems}

\item
Вершины выпуклого $n$-угольника ($n > 4$) расположены в~узлах сетки.
Докажите, что внутри многоугольника есть хотя~бы одна целая точка.

\item
Фигура~$S$ на~плоскости имеет площадь больше $1$.
Докажите, что найдутся такие точки $A, B \in S$, что вектор $\overline{AB}$
имеет целочисленные координаты.

\item
Внутри выпуклого стоугольника отмечено $k$ точек ($2 \leq k \leq 50$).
Докажите, что можно выбрать $2k$ вершин стоугольника так, что $2k$-угольник
с~вершинами них содержит все отмеченные точки.

\item
В~выпуклом многоугольнике на~плоскости содержится не~меньше $m^2 + 1$ точек
с~целыми координатами.
Докажите, что в~нем найдется $m + 1$ точек с~целыми координатами, которые лежат
на~одной прямой.

\end{problems}

\iffalse % BEGIN НА СВЕТЛОЕ БУДУЩЕЕ

\item
Есть несколько векторов на~плоскости с~суммой ноль.
Докажите, что их~можно расположить по~сторонам выпуклого многоугольника.

\fi % END НА СВЕТЛОЕ БУДУЩЕЕ

