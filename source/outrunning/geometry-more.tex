% $date: 2014-10-03

\section*{Добавка геометрии}

% $authors:
% - Андрей Кушнир

\begin{problems}

\item
$M$~--- середина стороны~$BC$ треугольника $ABC$.
Прямая, проходящая через $H$, перпендикулярная $HM$, пересекает стороны $AC$,
$AB$ в~точках $B_1$, $C_1$.
Докажите, что $H$~--- середина $B_1C_1$.

\item
В~треугольнике $ABC$ на~стороне~$BC$ отмечены точки $B_1$, $C_1$, так что
$\angle B A B_1 = \angle C A C_1$.
Докажите, что центры описанных окружностей треугольников $B A B_1$, $C A C_1$,
$B A C_1$, $C A B_1$ лежат на~одной окружности.

\item
$X$~--- точка пересечения диагоналей трапеции $ABCD$ ($AD \parallel BC$),
которые перпендикулярны.
На~прямой $AD$ отмечается точка $K$, затем описанные окружности треугольников
$AXK$ и~$DXK$ пересекаются с~прямыми $AB$, $CD$ в~точках $M$, $N$
соответственно.
А~доказать надо, что центр описанной окружности треугольника $MKN$ лежит
на~средней линии трапеции.

\end{problems}

