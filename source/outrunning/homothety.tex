% $date: 2014-12-17

\section*{Гомотетия}

\definition
Гомотетией с центром в точке $O$ и коэффициентом $k \neq 0$ называется
преобразование плоскости, которое каждую точку плоскости $X$ переводит
в точку~$X'$ так, что $\ov{OX'} = k \ov{OX}$.
Гомотетию с центром $O$ и коэффициентом $k$ обычно обозначают $H_O^k$. 

\claim{Важно}
Гомотетия переводит прямую в параллельную прямую, отрезок в отрезок, луч в луч,
окружность в окружность.
А также гомотетия переводит касающиеся объекты в касающиеся (пересекающиеся
в пересекающиеся). 

\claim{Теорема (о центрах гомотетий)}
Композиция двух гомотетий $H_{O_1}^{k_1}$ и $H_{O_2}^{k_2}$ является гомотетией
с коэффициентом $k_1 \cdot k_2$ и с центром лежащим на прямой $O_1 O_2$, если
$k_1 \cdot k_2 \neq 1$, и является параллельным переносом или тождественным
преобразованием, если $k_1 \cdot k_2 = 1$.

\begin{problems}

\item
На плоскости даны два неравных треугольника с параллельными соответственными
сторонами.
Докажите, что существует гомотетия, переводящая один треугольник в другой.

\item
На окружности фиксированы точки $A$ и $B$, а точка~$C$ движется по этой
окружности.
Найдите геометрическое место точек пересечения медиан треугольников $ABC$.

\item
В треугольнике $ABC$ проведена чевиана~$A A_1$.
Оказалось, что вписанные окружности треугольников $A A_1 B$ и $A A_1 C$ равны.
Докажите, что и вневписанные окружности этих треугольников, лежащие напротив
вершины~$A$, тоже равны.

\item
\sp
Дана окружность~$\omega$.
В сегмент, ограниченный хордой~$AB$, вписана окружность, касающаяся $\omega$
и~отрезка~$AB$ в точках $M$, $N$.
Докажите, что вторая точка пересечения прямой~$MN$ и окружности~$\omega$ делит
дугу~$AB$ пополам.
\\
\sp
Дана окружность~$\omega$.
Окружность~$s$ касается продолжения хорды~$AB$ в точке~$M$
и окружности~$\omega$ внешним образом в точке~$N$.
Докажите, что вторая точка пересечения прямой~$MN$ и окружности~$\omega$ делит
дугу~$AB$ пополам.

\item
Дан треугольник $ABC$.
Вписанная окружность касается стороны~$BC$ в точке~$K$.
Вневписанная окружность касается отрезка~$BC$ в точке~$L$.
Точка~$K'$ лежит на вписанной окружности и диаметрально противоположна
точке~$K$.
Точка~$L'$ лежит на вневписанной окружности и диаметрально противоположна
точке~$L$.
Докажите, что прямые $K'L$ и $KL'$ пересекаются в~точке~$A$.

\item
Каждая из окружностей $S_1$, $S_2$, $S_3$ касается внешним образом
окружности~$S$ (в точках $A_1$, $B_1$ и $C_1$ соответственно) и двух сторон
треугольника $ABC$, имеющих общую вершину $A$, $B$, $C$ соответственно.
Докажите, что прямые $A A_1$, $B B_1$ и $C C_1$ пересекаются в одной точке.

\item
Две окружности пересекаются в точках $A$ и $B$.
Через $A$ проводятся всевозможные прямые, вторично пересекающие окружности
в точках $M$ и $N$.
Докажите, что на плоскости существует точка, равноудаленная от $M$ и $N$ для
каждой пары этих точек.

%\item
%Внутри большой окружности нарисована маленькая.
%Окружности $\omega_1$, $\omega_2$, $\omega_3$, касаются большой окружности
%внутренним образом в точках $M_1$, $M_2$, $M_3$ и маленькой окружности внешним
%образом в точках $N_1$, $N_2$, $N_3$.
%Докажите, что $M_1 N_1$, $M_2 N_2$, $M_3 N_3$ пересекаются в одной точке.

\item
Внутри треугольника $ABC$ нарисованы четыре круга одинакового радиуса:
$\omega_A$, $\omega_B$, $\omega_C$ и $s$, причем каждый из кругов $\omega_i$
касается двух сторон треугольника и $s$.
Докажите, что центр круга $s$ принадлежит прямой, проходящей через центры
вписанной и описанной окружностей треугольника $ABC$.

%\item Докажите, что точки, симметричные произвольной точке относительно середин
%сторон квадрата,
%являются вершинами некоторого квадрата.

%\item Продолжения боковых сторон $AB$ и $CD$ трапеции $ABCD$ пересекаются в
%точке $K$, а её диагонали --- в точке $L$.
%Докажите, что точки $K$, $L$, $M$ и $N$, где $M$ и $N$ --- середины оснований
%$BC$ и $AD$, лежат на одной прямой.

%\item Окружность $S$ касается равных сторон $AB$ и $BC$ равнобедренного
%треугольника $ABC$ в точках $P$ и $K$, а также касается внутренним образом
%описанной окружности треугольника $ABC$.
%Докажите, что середина отрезка $PK$ является центром вписанной окружности
%треугольника $ABC$.

\item
В параллелограмме $ABCD$ на диагонали~$AC$ отмечена точка~$K$.
Окружность~$S_1$ проходит через точку~$K$ и касается прямых $AB$ и $AD$
($S_1$ вторично пересекает диагональ~$AC$ на отрезке~$AK$).
Окружность~$S_2$ проходит через точку~$K$ и касается прямых $CB$ и $CD$
($S_2$ вторично пересекает диагональ~$AC$ на отрезке~$KC$).
Докажите, что при всех положениях точки~$K$ на диагонали~$AC$ прямые,
соединяющие центры окружностей $S_1$ и $S_2$, будут параллельны между собой.
% (Т. Емельянова) Всероссийская олимпиада, окружной этап, 2000-2001 год, 10.2

\item
Пусть $AD$~--- биссектриса треугольника $ABC$, и прямая~$\ell$ касается
окружностей, описанных около треугольников $ADB$ и $ADC$ в точках $M$ и $N$
соответственно.
Докажите, что окружность, проходящая через середины отрезков $BD$, $DC$ и $MN$,
касается прямой~$\ell$.
% (Н.Седракян) Всероссийская олимпиада, окружной этап, 2000-2001 год, 11.3

\end{problems}

\iffalse % BEGIN МНОГО ЗАДАЧ

\item
а) Вписанная окружность треугольника $ABC$ касается стороны $AC$ в точке $D$,
$DM$ --- её диаметр.
Прямая $BM$ пересекает сторону $AC$ в точке $K$.
Докажите, что $AK = DC$.
б) В окружности проведены перпендикулярные диаметры $AB$ и $CD$.
Из точки $M$, лежащей вне окружности, проведены касательные к окружности,
пересекающие прямую $AB$ в точках $E$ и $H$, а также прямые $MC$ и $MD$,
пересекающие прямую $AB$ в точках $F$ и $K$.
Докажите, что $EF = KH$.

\item
Внутри угла $A$ выбрана произвольная точка $M$.
Постройте с помощью циркуля и линейки окружность, проходящую через $M$ и
касающуюся сторон угла $A$.

\item
В окружности $\omega$ проведена хорда $AB$. Окружность $\gamma$ касается $AB$ в точке $K$ и окружности $\omega$ в точке $T$ внутренним образом. $L$~--- середина дуги $AB$ окружности $\omega$, не содержащей точки $T$.
\sp Докажите, что точки $K$, $T$ и $L$ лежат на одной прямой ({\bf Лемма Архимеда}).
\sp Докажите, что степень точки $L$ относительно окружности $\gamma$ не зависит от выбора этой окружности.


\item
Докажите, что в любом треугольнике ортоцентр $H$, точка пересечения медиан $M$ и центр описанной окружности $O$ лежат на одной прямой причем $MH = 2MO$ ({\bf прямая Эйлера}).

\item
Обозначим точки касания вписанной и вневписанной окружностей со стороной $AC$ треугольника $ABC$ через $P$ и $Q$ соответственно.
Докажите, что прямая $BQ$ проходит через точку диаметрально противоположную точке $P$ на вписанной окружности.

\item
Внутри угла $A$ выбрана произвольная точка $M$.
Постройте с помощью циркуля и линейки окружность, проходящую через $M$ и касающуюся сторон угла $A$.

\item
Докажите, что любой выпуклый многоугольник $F$ содержит два непересекающихся во внутренних точках многоугольника $F_1$ и $F_2$, подобных $F$ с коэффициентом $\frac{1}{2}$.

\item
Даны две окружности $\omega_1$ и $\omega_2$. Окружность $\gamma$ касается их внешним образом в точках $A$ и $B$.
Докажите, что прямая $AB$ проходит через фиксированную точку, независящую от выбора $\gamma$.

\item
Окружность $\omega_A$ вписана в угол $A$ треугольника $ABC$. Аналогично определены окружности $\omega_B$ и $\omega_C$, причем все эти окружности не пересекаются. Окружность $\omega$ касается внешним образом окружностей $\omega_A$, $\omega_B$, $\omega_C$ в точках $A'$, $B'$ и $C'$ соответственно.
Докажите, что прямые $AA'$, $BB'$ и $CC'$ пересекаются в одной точке.

\item
\sp Ортоцентр $H$ треугольника $ABC$ симметрично отразили относительно середины стороны $AC$ и симметрично относительно стороны $AC$.
Докажите, что образы попали на описанную окружность треугольника.
\sp {\bf Окружность Эйлера.} Докажите, что середины сторон треугольника, основания высот треугольника и середины отрезков, соединяющих вершины треугольника с ортоцентром лежат на одной окружности.

\item
Из вершины $A$ треугольника $ABC$ проведен луч $AM$, лежащий внутри треугольника ($M$ лежит на $BC$). Обозначим через $\gamma_1$, $\gamma_2$ вписанная и вневписанная окружности треугольника $AMB$ соответственно (берется окружность, касающаяся стороны $MB$). Аналогично для треугольника $ACM$ определены окружности $\omega_1$, $\omega_2$.
Докажите, что общая внешняя касательная к окружностям $\gamma_1$ и $\omega_1$, отличная от $BC$ и общая внешняя касательная к окружностям $\gamma_2$ и $\omega_2$ отличная от $BC$ пересекаются на прямой $BC$.

\item
Внутри треугольника расположены окружности $\alpha $, $\beta $, $\gamma $, $\delta $ одинакового радиуса, причем каждая из окружностей $\alpha $, $\beta $, $\gamma $ касается двух сторон треугольника и окружности $\delta $.
Докажите, что центр окружности $\delta $ принадлежит прямой, проходящей через центры вписанной и описанной окружностей данного треугольника.

\item
Внутри выпуклого четырехугольника $ABCD$ выбрана точка $O$. Обозначим центр вписанной окружности треугольника $OAB$ через $\omega_{AB}$. Аналогично определим окружности $\omega_{BC}$, $\omega_{CD}$ и $\omega_{DA}$. Оказалось, что $\omega_{AB}$ касается $\omega_{BC}$, $\omega_{BC}$ касается $\omega_{CD}$, $\omega_{CD}$ касается $\omega_{DA}$ и $\omega_{DA}$ касается $\omega_{AB}$.
Докажите, что общая внешняя касательная к окружностям $\omega_{AB}$ и $\omega_{BC}$, общая внешняя касательная к окружностям $\omega_{CD}$ и $\omega_{DA}$, и прямая $AC$ пересекаются в одной точке. \\


В задачах этого листика (до задачи начинающейся так: "В четырехугольник $ABCD$ вписана окружность...") $\omega$, $\omega_{A}$, $\omega_{B}$, $\omega_{С}$ - вписанная и вневписанная окружности неравнобедренного треугольника $ABC$, $I$, $I_A$, $I_B$, $I_C$ - их центры; $K$, $L_A$, $L_B$, $L_C$ - точки их касания со стороной $BC$.

\item
\sp
Точка $K'$ лежит на $\omega$ и диаметрально противоположна точке $K$. Точка $L_A'$ лежит на $\omega_A$ и диаметрально противоположна точке $L_A$. Докажите, что прямые $KL_A'$ и $K'L_A$ пересекаются в точке $A$.
\sp
Точка $L_B'$ лежит на $\omega_B$ и диаметрально противоположна точке $L_B$. Точка $L_C'$ лежит на $\omega_C$ и диаметрально противоположна точке $L_C$. Докажите, что прямые $L_BL_C'$ и $L_B'L_C$ пересекаются в точке $A$.

\item
\begin{enumerate}
\item
Из середины $BC$ провели вторые касательные к $\omega$ и $\omega_A$, точки касания обозначили $R$ и $R_A$. Докажите, что прямые $RL_A$ и $R_AK$ пересекаются в точке $A$.
\item
Из середины $BC$ провели вторые касательные к $\omega_B$ и $\omega_C$, точки касания обозначили $R_B$ и $R_C$. Докажите, что прямые $R_BL_C$ и $R_CL_B$ пересекаются в точке $A$.
\end{enumerate}

\item
\begin{enumerate}
\item
Докажите, что $I_AK$ и $IL_A$ пересекаются на середине высоты $AA_1$.
\item
Докажите, что $I_BL_C$ и $I_CL_B$ пересекаются на середине высоты $AA_1$.
\end{enumerate}

\item
$I_A$, $I_B$, $I_C$ соединили прямыми с серединами высот $AA_1$, $BB_1$, $CC_1$ соответственно. Докажите, что проведенные прямые пересекаются в одной точке.

\item
В четырехугольник $ABCD$ вписана окружность с центром $I$. Лучи $AB$ и $DC$ пересекаются в точке $X$. Вписанная в треугольник $XBC$ окружность касается отрезка $BC$ в точке $P$. Вневписанная в треугольник $XAD$ окружность касается отрезка $AD$ в точке $Q$. Известно, что прямая $PQ$ проходит через $A$. Докажите, что отрезок, соединяющий середины сторон $BC$ и $AD$, проходит через точку $I$.

\fi % END МНОГО ЗАДАЧ

