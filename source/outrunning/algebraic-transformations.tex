% $date: 2015-01-23

\section*{Алгебраические преобразования}

\begin{problems}

\item
Для некоторых натуральных чисел $a$, $b$, $c$, $d$ выпполняются равенства
\[
    \frac{a}{c} = \frac{b}{d} = \frac{a b + 1}{c d + 1}
\;.\]
Докажите, что $a = c$ и $b = d$.
%2002-2003, зона 8.6

\item
Докажите тождество
\begin{gather*}
    \frac{a_1}{a_2 (a_1 + a_2)} +
    \frac{a_2}{a_3 (a_2 + a_3)}
    + \ldots +
    \frac{a_n}{a_1 (a_n + a_1)}
=\\=
    \frac{a_2}{a_1 (a_1 + a_2)} +
    \frac{a_3}{a_2 (a_2 + a_3)}
    + \ldots +
    \frac{a_1}{a_n (a_n + a_1)}
\;.\end{gather*}


\item
Рациональные числа $a$ и $b$ удовлетворяют соотношению
\[
    a^3 b + a b^3 + 2 a^2 b^2 + 2 a + 2 b + 1
=
    0
\,.\]
Докажите, что $1 - a b$ является квадратом рационального числа.

%\item
%Докажите, что $x \cdot \cos x \leq \frac{\pi^2}{16}$ при
%$0 \leq x \leq \frac{\pi}{2}$.

\item
Существует~ли такое действительное $\alpha$, что $\cos (\alpha)$ иррационально,
а числа $\cos (2 \alpha)$, $\cos (3 \alpha)$, $\cos (4 \alpha)$ и
$\cos (5 \alpha)$ рациональны?
% 11.1

%\item
%Ненулевые числа $a$ и $b$ удовлетворяют равенству
%\[
%    a^2 b^2 (a^2 b^2 + 4) = 2 (a^6 + b^6)
%\,.\]
%Докажите, что хотя бы одно из них иррационально.

\item
Числа $a$ и $b$ таковы, что $a^3 - b^3 = 2$ и $a^5 - b^5 \geq 4$.
Докажите, что $a^2 + b^2 \geq 2$.
% 9.6

\item
Числа $x$, $y$, $z$ таковы, что $x + y z$, $y + x z$, $z + x y$ рациональны,
а $x^2 + y^2 = 1$.
Докажите, что число $x y z^2$ рационально.
% 11.5

\item
Найдите все такие натуральные $n$, что при некоторых различных натуральных
$a$, $b$, $c$ и $d$ среди чисел
\[
    \frac{(a - c) (b - d)}{(b - c) (a - d)}
,\;
    \frac{(b - c) (a - d)}{(a - c) (b - d)}
,\;
    \frac{(a - b) (d - c)}{(a - d) (b - c)}
,\;
    \frac{(a - c) (b - d)}{(a - b) (c - d)}
\]
есть по крайней мере два числа, равных $n$.
%1995-1996, зона 11.6

\item
Число~$N$, не делящееся на $81$, представимо в виде суммы квадратов трёх целых
чисел, делящихся на~$3$.
Докажите, что оно также представимо в виде суммы квадратов трёх целых чисел,
не делящихся на $3$.

\end{problems}

