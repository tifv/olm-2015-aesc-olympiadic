% $date: 2015-01-14

% $previous-dates:
% - 2011-03-06
% - 2013-03-06

\section*{Тренировочная олимпиада, 10 класс}

% $build$matter[print]: [[.], [.], [.], [.]]
% $build$style[print]:
% - .[tiled4,-print]

% $style:
% - /

\begin{problems}

\item
На~какое наименьшее число равновеликих треугольников можно разрезать фигуру,
получаемую из~квадрата $8 \times 8$ вырезанием угловой клетки $1 \times 1$?

\item
Дано уравнение
$x^n - a_1 x^{n-1} - a_2 x^{n-2} - \ldots - a_n = 0$,
где $a_1 > 0$, $a_2 > 0$, $\ldots$, $a_n > 0$.
Какое наибольшее количество положительных корней может быть у~этого уравнения?

\item
На~стороне $AC$ остроугольного треугольника $ABC$ выбраны точки $M$ и~$K$ так,
что $\angle ABM = \angle CBK$.
Докажите, что центры окружностей, описанных около треугольников
$ABM$, $ABK$, $CBM$ и~$CBK$, лежат на~одной окружности.

\item
Два мудреца играют в~следующую игру.
Выписаны числа $0, 1, 2, \ldots, 1024$.
Первый мудрец зачеркивает $512$ чисел (по~своему выбору), второй зачеркивает
$256$ из~оставшихся, затем снова первый зачеркивает $128$ чисел и~т.~д.
На~десятом шаге второй мудрец зачеркивает одно число; остаются два числа.
После этого второй мудрец платит первому разницу между этими числами.
Как выгоднее играть первому мудрецу?
Как второму?
Сколько уплатит второй мудрец первому, если оба будут играть наилучшим образом?

\item
Пусть $a_r$~--- количество полных квадратов, содержащихся в~$r$-й тысяче,
т.\,е. в~%промежутке
$[1000 (r - 1), 1000r)$.
Докажите, что последовательность $a_r$ не~является периодической ни~с~какого
номера.

\end{problems}

