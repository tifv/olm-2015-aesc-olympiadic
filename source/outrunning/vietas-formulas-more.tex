% $date: 2014-09-19

\section*{Теорема Виета. Добавка}

% $authors:
% - Олег Орлов

% $build$matter[print]: [[.], [.]]

\definition
Многочлен от $n$~переменных $P(x_1, x_2, \ldots, x_n)$ называется
\emph{симметрическим}, если он~не~изменяется при всех перестановках переменных
$x_1, \ldots, x_n$.

\definition
\begin{align*} &
    \sigma_1 = x_1 + \ldots + x_n
,\\ &
    \sigma_2 = x_1 x_2 + x_1 x_3 + \ldots + x_{n-1} x_n
    \text{\quad(сумма всех попарных произведений)}
,\\ & \ldots \\ &
    \sigma_{n-1}
=
    x_1 x_2 \cdots x_{n-1} + x_1 x_2 \cdots x_{n-2} x_n
    + \ldots +
    x_2 x_3 \cdots x_n
,\\ &
    \sigma_n = x_1 x_2 \cdots x_n
.\end{align*}
Многочлены $\sigma_k$ называются \emph{основными (элементарными)}
симметрическими многочленами.

\theoremof{Виета}
Пусть многочлен $a_n x^n + a_{n-1} x^{n-1} + \ldots a_1 x + a_0$ имеет
$n$~корней (с~учетом кратности):
$x_1, x_2, \ldots, x_n$.
Тогда $\sigma_k = (-1)^k \frac{a_{n-k}}{a_n}$, для любого $1 \leq k \leq n$.

\claim{Основная теорема о симметрических многочленах}
Пусть $P(x_1, \ldots, x_n)$~--- симметрический многочлен от~$n$~переменных,
тогда его единственным образом можно представить в~виде
$P(x_1, \ldots, x_n) = g(\sigma_1, \ldots, \sigma_n)$, где $g$~--- многочлен,
а~$\sigma_k$~--- элементарные симметрические многочлены от~$n$ переменных.

\begin{problems}

\item
Пусть действительные числа $a, b, c$ таковы, что
\[
    \frac{1}{a} + \frac{1}{b} + \frac{1}{c}
=
    \frac{1}{a + b + c}
.\]
Докажите, что для любого нечетного натурального $k$ выполняется равенство:
\[
    \frac{1}{a^k} + \frac{1}{b^k} + \frac{1}{c^k}
=
    \frac{1}{a^k + b^k + c^k}
.\]

\item
\sp\emph{Формула Ньютона.}
Докажите, что
\[
    S_k
=
    S_{k-1} \sigma_{1} - S_{k-2} \sigma_{2} + \ldots + (-1)^k S_0 \sigma_k
,\]
где $S_k = x_1^k + \ldots + x_n^k$, а~$\sigma_k$~--- элементарные
симметрические многочлены от~$n$~переменных $x_1, x_2, \ldots, x_n$.
\\
\sp
Пусть имеются два упорядоченных набора чисел
$x_1 < x_2 < \ldots < x_7$ и~$y_1 < y_2 < \ldots < y_7$.
Известно, что $x_1 < y_1$ и~суммы $k$-ых степеней равны для всех $k$ от~1 до~6.
Докажите, что $x_7 < y_7$.

\item
Пусть $x_1, \ldots, x_n$~--- корни многочлена
$x^n + x^{n - 1} + \ldots + x + 1$.
Докажите, что
\[
    \frac{1}{1 - x_1} + \ldots + \frac{1}{1 - x_n}
=
    \frac{n}{2}
.\]

\end{problems}

