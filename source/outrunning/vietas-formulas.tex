% $date: 2014-09-17

\section*{Теорема Виета}

% $authors:
% - Глеб Погудин

% $build$matter[print]: [[.], [.]]

\begin{problems}

\item
Числа $x, y, z$ удовлетворяют системе:
\[ \left\{ \begin{aligned} &
    x  + y  + z  = a
\\ &
    \frac{1}{x} + \frac{1}{y} + \frac{1}{z} = \frac{1}{a}
\end{aligned} \right. \]
Докажите, что хотя~бы одно из~них равно $a$.

\item
На~параболе $y = x^2$ выбраны четыре точки $A$, $B$, $C$, $D$ так, что прямые
$AB$ и~$CD$ пересекаются на~оси ординат.
Найдите абсциссу точки $D$, если абсциссы точек $A$, $B$ и~$C$ равны
$a$, $b$ и~$c$ соответственно.

\item
Известно, что целые числа $a$, $b$, $c$ удовлетворяют равенству
$a + b + c = 0$.
Докажите, что $2 a^4 + 2 b^4 + 2 c^4$~--- квадрат целого числа.

\item
Существуют~ли такие ненулевые числа $a$, $b$, $c$, что при любом $n > 3$ можно
найти многочлен вида $p_n(x) = x^n + \ldots + a x^2 + b x + c$, имеющий ровно
$n$ (не~обязательно различных) целых корней?

%\item
%Числа $a, b, c$ рациональны.
%Про корни $x_1, x_2, x_3$ многочлена известно, что $\frac{x_1}{x_2}$
%рационально, не~равно нулю и~минус единице.
%Докажите, что все три числа $x_1$, $x_2$ и~$x_3$ рациональны.

\item
Пусть имеются два упорядоченных набора чисел $x_1 < x_2 < \ldots < x_7$
и~$y_1 < y_2 < \ldots < y_7$.
Известно, что $x_1 < y_1$ и~суммы $k$-ых степеней равны для всех $k$ от~1 до~6.
Докажите, что $x_7 < y_7$.

%\item
%Пусть $x_1, \ldots, x_n$~--- корни многочлена
%$x^n + x^{n - 1} + \ldots + x + 1$.
%Докажите, что
%$$ \frac{1}{1 - x_1} + \ldots + \frac{1}{1 - x_n} = \frac{n}{2}.$$

\item
Про действительные числа $a \leq b \leq c$ известно, что $a + b + c = 2$
и~$a b + b c + c a = 1$.
Докажите, что
$0 \leq a \leq \frac{1}{3} \leq b \leq 1 \leq c \leq \frac{4}{3}$.

\end{problems}

