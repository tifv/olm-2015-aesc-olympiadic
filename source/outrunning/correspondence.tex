% $date: 2014-10-15

\section*{Соответствия}

% $authors:
% - Олег Орлов

\begin{problems}

\item
Дан выпуклый $n$-угольник такой, что никакие три его диагонали не~пересекаются
в~одной точке.
Найдите количество точек пересечения диагоналей данного многоугольника
(не~являющиеся вершинами многоугольника).

\item
На~клетчатой бумаге изображен квадрат, сторона которого умещает ровно
$n$~клеток.
Сколько в~этом квадрате можно уместить различных
\\
\sp квадратов?
\qquad
\sp прямоугольников?
\\
\sp
букв <<Г>> (в~том числе и~как угодно перевернутых)?
Здесь буква <<Г>>~--- это объединение двух прямоугольников
$1 \times n$ и~$m \times 1$, пересекающихся своими концами, где $m, n > 1$.
\\
(Стороны фигур проходят по~сторонам сетки.)

\item
Докажите, что количество разбиений числа $n$ в~сумму не~более чем
$k$ слагаемых, равно количеству разбиений числа $n$ в~сумму слагаемых,
не~превосходящих $k$.
% хотелось бы дать эту задачу, чтобы продемонстрировать вездесущесть диаграмм
% Юнга. В листочке с двойным подсчетом первая тоже была на Юнга, я говорил про
% них на разборе. (Глеб)

\item
Докажите, что количество разбиений числа $n$ в~сумму различных слагаемых равно
количеству разбиений числа $n$ в~сумму нечетных слагаемых.

\item
\sp
Рассмотрим набор чисел $\{a_1, \ldots, a_{2n + 1}\}$, где каждое равно
$\pm 1$ и~сумма всех чисел набора равна единице.
Докажите, что набор можно циклически сдвинуть так, что все частичные суммы
будут положительны.
\\
\sp
Сколько последовательностей  $\{a_1, a_2, \ldots , a_{2n}\}$, состоящих
из~единиц и~минус единиц, обладают тем свойством, что 
$a_1 + a_2 + \ldots + a_{2n} = 0$,  а~все частичные суммы
$a_1,  a_1 + a_2, \ldots,  a_1 + a_2 + \ldots + a_{2n}$ неотрицательны?
\\
\sp
Сколько существует способов расставить скобки в~произведении
$x_0 \cdot x_1 \cdot \ldots \cdot x_n$ так, чтобы порядок умножений был
полностью определен?

\item
Рассмотрим последовательность из~$n$ натуральных чисел.
Будем называть её~\emph{уморительной}, если вместе с~каждым $k \geqslant 2$
в~последовательность входит также и~число $k - 1$, причем первое вхождение
$k - 1$ до~последнего вхождения $k$.
Сколько уморительных последовательностей существует?

\end{problems}

