% $date: 2014-12-03

\section*{Степень точки. Радикальные оси}

\definition
Пусть даны точка~$P$ и~окружность~$\omega$.
Тогда \emph{степенью точки $P$} относительно окружности~$\omega$ называется
число $d^2 - R^2$, где $d$~--- это расстояние от~точки~$P$ до~центра
окружности~$\omega$, а~$R$~--- её~радиус
(степень точки может быть отрицательной).

\claim{Утверждение 1}\setcounter{jeolmsubproblem}{0}
\sp
Пусть точка~$P$ лежит вне окружности $\omega$, и~через точку~$P$ проведены
две прямые, одна из~которых касается окружности в~точке~$Q$, а~другая
пересекает окружность в~точках $A$ и~$B$.
Тогда степень точки~$P$ относительно окружности~$\omega$ равна
$PA \cdot PB = PQ^2$.
\\
\sp
Пусть точка~$P$ лежит внутри окружности $\omega$, и~через точку~$P$ проведена
прямая, пересекающая окружность в~точках $A$ и~$B$.
Тогда степень точки~$P$ относительно окружности~$\omega$ равна
$(- PA \cdot PB)$.

\claim{Утверждение 2}
Пусть даны две не~концентрические окружности $\omega_1$ и~$\omega_2$.
Тогда геометрическое место точек~$P$, для которых степени относительно обеих
окружностей равны, будет прямой, которая перпендикулярна линии центров
окружностей $\omega_1$ и~$\omega_2$.

\definition
Полученная прямая называется \emph{радикальной осью} окружностей $\omega_1$
и~$\omega_2$.

\claim{Утверждение 3}
Для трех окружностей $\omega_1$, $\omega_2$, $\omega_3$, центры которых
не~лежат на~одной прямой, существует единственная точка такая, что её~степени
относительно всех трёх окружностей равны.

\definition
Такая точка называется \emph{радикальным центром} окружностей
$\omega_1$, $\omega_2$ и~$\omega_3$.


\subsection*{Задачи}

\begin{problems}

\item
Докажите утверждения 1 и~3 (утверждением~2 можно пользоваться).

\item
На~гипотенузе~$AB$ прямоугольного равнобедренного треугольника $ABC$
выбрана произвольная точка~$M$.
Докажите, что общая хорда окружностей с~центром~$C$ и~радиусом~$CA$
и~с~центром~$M$ и~радиусом~$MC$ проходит через середину $AB$.

\item
Постройте окружность, проходящую через две данные точки и~касающуюся данной
прямой.

\item
Докажите, что середины отрезков всех общих касательных к~двум непересекающимся
окружностям лежат на~одной прямой.

\item
Пусть $B_1$ и~$C_1$~--- точки касания вписанной окружности треугольника $ABC$
со~сторонами $AC$ и~$AB$.
На~продолжениях сторон $AB$ и~$AC$ за~точки $B$ и~$C$ отметили точки $X$, $Y$
соответственно так, что $C_1 X = B_1 Y = BC$.
Докажите, что середины отрезков $C_1 X$, $B_1 Y$ и~$BC$ лежат на~одной прямой.

\item
На~сторонах $BC$ и~$AC$ треугольника $ABC$ выбраны точки $A_1$ и~$B_1$
соответственно.
Докажите, что прямая, проходящая через точки пересечения окружностей,
построенных на~отрезках $A A_1$ и~$B B_1$ как на~диаметрах, проходит через
ортоцентр треугольника $ABC$.

\item
Пусть продолжения сторон $AB$ и~$CD$ четырехугольника $ABCD$ пересекаются
в~точке~$P$, а~продолжения сторон $BC$ и~$AD$ в~точке $Q$.
\\
\sp
Докажите, что ортоцентры треугольников $BPC$, $APD$, $ABQ$ и~$CDQ$ лежат
на~одной прямой.
\\
\sp\emph{Прямая Гаусса.}
Докажите, что середины $AC$, $BD$ и~$PQ$ лежат на~одной прямой.

\item
$A_1$, $B_1$, $C_1$~--- точки касания вписанной в~треугольник $ABC$ окружности
со~сторонами $BC$, $CA$, $AB$ соответственно.
Точка~$P$~--- произвольная.
Серединный перпендикуляр к~отрезку~$P A_1$ пересекает прямую~$BC$
в~точке~$A_2$.
Угадайте как строятся точки $B_2$, $C_2$.
Докажите, что $A_2$, $B_2$, $C_2$ лежат на~одной прямой.

\item
\sp
Через точку~$P$, лежащую на~общей хорде~$AB$ двух пересекающихся окружностей,
проведены хорда $A_1 B_1$ первой окружности и~хорда $A_2 B_2$
второй окружности.
Докажите, что четырехугольник $A_1 A_2 B_1 B_2$~--- вписанный.
\\
\sp\emph{Теорема о бабочке.}
Через середину~$P$ хорды~$AB$ окружности проведены секущие $A_1 A_2$
и~$B_1 B_2$.
Хорды $A_1 B_1$ и~$A_2 B_2$ пересекают хорду~$AB$ в~точках $M$ и~$N$.
Докажите, что $PM = PN$.

\end{problems}


\iffalse % BEGIN НА СВЕТЛОЕ БУДУЩЕЕ

%задачи никак не упорядочены пока О.

\item
На~сторонах треугольника $ABC$ во~внешнюю границу, как на~основаниях, построены
равнобедренные треугольники $BCD$, $CAE$ и~$ABF$.
Докажите, что прямые, проходящие через точки $A$, $B$ и~$C$ перпендикулярно
$EF$, $FD$ и~$DE$ соответственно, пересекаются в~одной точке.
% +

\item
В~угол вписаны две окружности.
Одна окружность касается одной стороны угла в~точке~$A$, вторая окружность
касается другой стороны угла в~точке~$B$.
Докажите, что прямая~$AB$ высекает на~окружностях равные хорды.
% подсчёт степени точки
% +

\item
С~центром в~точке~$O$ построены большая окружность и~маленькая окружность.
Из~точки~$A$ большой окружности проведены касательные $AB$, $AC$ к~маленькой;
$B$, $C$~--- точки касания.
Окружность с~центром~$А$ и~радиусом~$AB$ пересекает большую окружность
в~точках $M$ и~$N$.
Докажите, что прямая~$MN$ содержит среднюю линию треугольника $ABC$.
% подсчёт степени точки
% + !

\item
Серединный перпендикуляр к~стороне~$AC$ треугольника $ABC$ пересекает
прямые $BA$ и~$BC$ в~точках $B_1$ и~$B_2$.
Серединный перпендикуляр к~стороне~$AB$ пересекает прямые $CA$ и~$CB$
в~точках $C_1$ и~$C_2$.
Описанные окружности пересекаются в~точках $P$ и~$Q$.
Докажите, что центр описанной окружности треугольника $ABC$ лежит
на~прямой~$PQ$.
% подсчёт степени точки
% +

\item
На~боковых сторонах трапеции как на~диаметрах построены окружности.
Докажите, что отрезки касательных, проведенные к~этим окружностям из~точки
пересечения диагоналей, равны.
% подсчёт степени точки
% +

\item
На~окружности $\omega_1$ с~диаметром~$AB$ взята точка~$C$, из~точки~$C$ опущен
перпендикуляр~$CH$ на~прямую $AB$.
Докажите, что общая хорда окружности $\omega_1$ и~окружности $\omega_2$
с~центром~$C$ и~радиусом~$CH$ делит отрезок~$CH$ пополам.
% неприятный?? подсчёт степени точки
% +

\item
Точка~$M$~--- середина хорды~$AB$ некоторой окружности.
Хорда $CD$ той~же окружности пересекает $AB$ в~точке~$M$.
На~отрезке~$CD$ как на~диаметре построена полуокружность.
Точка~$E$ лежит на~этой полуокружности, причем~$ME$~--- перпендикуляр к~$CD$.
Найдите угол $AEB$.
% извлекаем свойства
% +

\item
Докажите, что в~описанном четырехугольнике точка пересечения хорд вписанной
окружности, соединяющих точки касания с~противоположными сторонами, является
также точкой пересечения диагоналей четырехугольника.
% достраиваем две окружности, которых нет в условии задачи

\item
Докажите, что главные диагонали описанного шестиугольника пересекаются в~одной
точке.
% достраиваем три окружности, которых нет в условии задачи
% нереально догадаться, если не знаешь

\item
Из~центра вписанной в~треугольник $ABC$ окружности параллельно высоте~$AH$
выпустили прямую, пересекающую медиану~$AM$ в~точке~$K$.
Докажите, что точка~$K$ лежит на~хорде вписанной окружности, соединяющей точки
её~касания со~сторонами $AB$ и~$AC$.
% красиво решается (Зайцевой Т.) через теорему Брианшона:
% проводим параллельную стороне $BC$ касательную к вписанной окружности,
% заменяем в условии точку пересечения хорд на точку пересечения диагоналей.
% медиана достроенного треугольника проходит через точку пересечения диагоналей
% трапеции
% ещё есть решение:
% рассмотрим гомотетию с центром $A$, переводящую $I$ в середину дуги $BC$.

\fi % END НА СВЕТЛОЕ БУДУЩЕЕ

