% $date: 2014-10-17

\section*{Соответствия и двойной подсчёт. Добавка}

% $authors:
% - Олег Орлов

\begin{problems}

\item
Можно~ли расставить по~кругу $7$ целых неотрицательных чисел так, чтобы сумма
каких-то трех расположенных подряд чисел была равна~$1$,
каких-то трех подряд расположенных~---~$2$, $\ldots$,
каких-то трех подряд расположенных~---~$7$?

\item
Может~ли оканчиваться на~$3$ сумма делителей числа (считая единицу и~само
число), оканчивающегося на~$3$?

\item
По~кругу расставлены красные и~синие числа.
Каждое красное число равно сумме соседних чисел, а~каждое синее~--- полусумме
соседних чисел.
Докажите, что сумма красных чисел равна нулю.

\item
При посадке в~самолет выстроилась очередь из~$n$~пассажиров, у~каждого
из~которых имеется билет на~одно из~$n$~мест (на~каждое место есть ровно
по~одному билету).
Первой в~очереди стоит сумасшедшая старушка.
Она вбегает в~салон и~садится на~случайное место (возможно, и~на~свое).
Далее пассажиры по~очереди занимают свои места, а~в~случае, если свое место уже
занято, садятся случайным образом на~одно из~свободных мест.
Какова вероятность того, что последний пассажир займет свое место?

\end{problems}

