% $date: 2015-01-16

\section*{Выход в пространство}

% $required$packages:
% - wrapfig

\begin{wrapfigure}{L}{0.4\textwidth}
\vspace{-1em}
\jeolmfigure[width=\linewidth]{desargues}
\vspace{-2em}
\end{wrapfigure}
\problem\emph{Теорема Дезарга.}
Если два треугольника $A_1 B_1 C_1$ и~$A_2 B_2 C_2$ расположены на~плоскости
так, что прямые $A_1 A_2$, $B_1 B_2$ и~$C_1 C_2$ пересекаются в~одной точке,
то~три точки пересечения прямых $A_1 B_1$ и~$A_2 B_2$, $B_1 C_1$ и~$B_2 C_2$,
$A_1 C_1$ и~$A_2 C_2$ лежат на~одной прямой.

\solution
Заметим на~рисунке пространственную фигуру.
Рассмотрим трёхгранный угол с~вершиной в~точке~$X$, который проецируется
в~точности в~наш рисунок (можно например, оставив на~месте точку~$X$,
приподнять луч~$X A_1$ вместе с~точками $A_1$ и~$A_2$, тогда
точки $D$, $E$ и~$F$, определяемые как пересечения соответствующих условию
прямых, тоже отойдут от~плоскости так, что их~проекциями будут
точки $D$, $E$, $F$ на~плоскости рисунка).
Осталось заметить, что точки $D$, $E$ и~$F$ принадлежат одновременно плоскостям
$A_1 B_1 C_1$ и~$A_2 B_2 C_2$, пересечением которых является прямая
(совпадать плоскости не~могут, если $X A_1 B_1 C_1$~--- трехгранный угол),
т.~е. точки $D$, $E$ и~$F$ лежат на~одной прямой в~пространстве, а~значит
и~лежат на~одной прямой после проекции.

\begin{problems}

\item
На~плоскости даны три параллельные прямые $a,b,c$ и~три точки $A,B,C$, лежащие
между прямыми $b$ и~$c$, $a$ и~$c$, $a$ и~$b$ соответственно.
Существует~ли треугольник такой, что его вершины лежат
на~прямых $a$, $b$ и~$c$, а~стороны содержат точки $A$, $B$, $C$
(на~каждой из~прямых $a$, $b$, $c$ должно быть не~более одной вершины
треугольника, и~на~каждой стороне не~более одной точки $A$, $B$, $C$).
% Нужно приподнять одну прямую 

\item\emph{Теорема Брианшона.}
Диагонали, соединяющие противоположные вершины описанного шестиугольника,
пересекаются в~одной точке.
\\
\sp
Пусть $ABCDEF$~--- данный описанный шестиугольник.
Докажите, что существует пространственный шестиугольник, проходящий через точки
касания $ABCDEF$ с~его вписанной окружностью, проекцией которого
на~плоскость $ABC$ будет шестиугольник $ABCDEF$ (пространственным
многоугольником назовём замкнутую несамопересекающуюся ломаную в~пространстве.
В~задаче требуется найти пространственный шестиугольник, не~лежащий в~одной
плоскости).
\\
\sp
Докажите теорему Брианшона.

% A_1 --- ставим произвольно, потом строим.
% A_1 B_1 C_1 D_1 E_1 F_1 --- полученный пространственный шестиугольник, тогда
% A_1 B_1 \parallel D_1 E_1

\item\emph{Теорема о~трёх колпаках.}
На~плоскости даны три непересекающиеся окружности.
Рассмотрим три точки пересечения общих внешних касательных к~каким-то~двум
из~данных окружностей.
Докажите, что эти точки лежат на~одной прямой.
% На окружностях строим сферы как на центральных сечениях.
% Все наши точки будут лежат на пересечении касающихся (внешним образом)
% плоскостей к сферам (каждая наша точка --- вершина одного из трех конусов).

\item
На~плоскости даны четыре прямые общего положения.
По~каждой прямой с~постоянной скоростью идёт пешеход.
Известно, что первый встречается со~вторым, с~третьим и~с~четвёртым, а~второй
встречается с~третьим и~с~четвёртым.
Доказать, что третий пешеход встретится с~четвёртым.
% Вводим третью ось --- ось времени.

\item
Через центр правильного треугольника $ABC$ провели произвольную прямую~$l$,
пересекающую стороны $AB$ и~$BC$ в~точках $D$ и~$E$.
Построили точку~$F$ такую, что $AE = FE$ и~$CD = FD$.
Докажите, что расстояние от~точки~$F$ до~прямой~$l$ не~зависит от~выбора этой
прямой.
% Мат. Многоборье --- 2015, командная-старшие-6. (М. Волчкевич)
% нужно достроить до правильного тетраэдра

%\item
%Внутри круглого блина радиуса 10 запекли монету радиуса 1.
%Каким наименьшим числом прямолинейных разрезов можно наверняка задеть монету? 
% Ответ 10.
% Оценка из задачи:
% Назовем слоем ширины d часть пространства между двумя параллельными
% плоскостями, находящимися на расстоянии d.
% Тогда сферу радиуса 10 нельзя покрыть меньше чем 10 слоями ширины 2.
% Используется тот факт, что площадь шарового слоя (часть сферы, отсеченная
% слоем) равна 2 \pi R h, где R --- радиус сферы, h --- высота слоя
% (это школьная программа же?).
% Абсолютная гробина, но красиво)

\item
Докажите, что с~помощью одной линейки нельзя найти центр окружности.
% Пусть наша окружность в плоскости α.
% Если существуют плоскость β и точка O такие, что проекция через точку O
% окружности на α является окружностью на β, причём центр не переходит в центр,
% то мы решили задачу.
% Тоже очень сложная, но тоже красиво)

% Две последние конечно неподъёмные (мне кажется так, но я сразу решение читал,
% так что необъективно), но даже послушать им будет просто интересно.

\end{problems}

