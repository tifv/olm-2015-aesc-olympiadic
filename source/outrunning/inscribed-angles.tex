% $date: 2014-10-01

\section*{Вписанные углы}

% $authors:
% - Андрей Кушнир

\begin{problems}

\item
Две окружности пересекаются в~точках $P$ и~$Q$.
Третья окружность с~центром~$P$ пересекает первую окружность в~точках
$A$ и~$B$, а~вторую~--- в~точках $C$ и~$D$.
Докажите, что $\angle AQD = \angle BQC$.

\item
Из~точки~$M$, двигающейся по~окружности, опускаются перпендикуляры $MP$ и~$MQ$
на~диаметры $AB$ и~$CD$.
Докажите, что длина отрезка $PQ$ не~зависит от~положения точки~$M$.

\item
Дан равнобедренный треугольник $ABC$ ($AB = AC$).
На~меньшей дуге~$AB$ описанной около него окружности взята точка $D$.
На~продолжении отрезка~$AD$ за~точку~$D$ выбрана точка~$E$ так, что точки $A$
и~$E$ лежат в~одной полуплоскости относительно $BC$.
Описанная окружность треугольника $BDE$ пересекает сторону $AB$ в~точке~$F$.
Докажите, что прямые $EF$ и~$BC$ параллельны.

%\item
%Окружность~$S_1$ касается сторон угла $\angle ABC$ в~точках $A$ и~$C$.
%Окружность~$S_2$ касается прямой~$AC$ в~точке~$C$ и~проходит через точку~$B$;
%окружность~$S_1$ она пересекает в~точке~$M$.
%Докажите, что прямая~$AM$ делит отрезок~$BC$ пополам.

\item
Докажите, что в~остроугольном треугольнике середины двух высот, основание
третьей и~ортоцентр лежат на~одной окружности.

\item
На~диагонали $AC$ ромба $ABCD$ взята произвольная точка $E$, отличная от~точек
$A$ и~$C$, а~на~прямых $AB$ и~$BC$~--- точки $N$ и~$M$ соответственно так, что
$AE = NE$ и~$CE = ME$.
Пусть $K$~--- точка пересечения прямых $AM$ и~$CN$.
Докажите, что точки $K$, $E$ и~$D$ лежат на~одной прямой.

\item
Дан выпуклый шестиугольник $ABCDEF$.
Известно, что $\angle FAE = \angle BDC$, а~четырехугольники $ABDF$ и~$ACDE$
являются вписанными.
Докажите, что прямые $BF$ и~$CE$ параллельны.

\end{problems}

