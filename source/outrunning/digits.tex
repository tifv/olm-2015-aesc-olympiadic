% $date: 2015-01-21

\section*{Циферки}

\begin{problems}

\item
Натуральное число $m$ таково, что сумма цифр числа $8^m$ равна $8$.
Может~ли $8^m$ оканчиваться на~$6$?
% 9.6

\item
Можно~ли при каком-то~натуральном $k$ разбить все натуральные числа от~$1$
до~$k$ на~две группы и~выписать числа в~каждой группе подряд в~некотором
порядке так, чтобы получились два одинаковых числа?
% 9.3

%\item
%Даны различные натуральные числа $a_1, \ldots, a_{14}$.
%На~доску выписаны все $196$ чисел вида $a_k + a_l$, где $1 \leq k, l \leq 14$.
%Может~ли оказаться, что для любой комбинации из~двух цифр среди написанных
%на~доске чисел найдется хотя~бы одно число, оканчивающееся на~эту комбинацию
%(то~есть, найдутся числа, оканчивающиеся на~$00, 01, 02, \ldots, 99$)?
%% 10.3

\item
Даны натуральные числа $M$ и~$N$, большие десяти, состоящие из~одинакового
количества цифр.
Кроме того, $M = 3 N$.
Чтобы получить число~$M$, надо к~одной из~цифр $N$ прибавить $2$, а~ко~всем
остальным цифрам по~нечетной цифре.
Какой цифрой могло оканчиваться число~$N$?
Найти все варианты.
% 9.1

%\item
%Три натуральных числа таковы, что последняя цифра суммы любых двух равна
%последней цифре третьего.
%На~какие три цифры могло заканчиваться произведение всех этих трех чисел?
%Найдите все варианты.
%% 11.1

%%%%%%%%%%%%%%%%%%%%%%%%%%%%%%%%%%%%%%%%

\item
На~экране компьютера горит число, которое каждую минуту увеличивается на~$102$.
Начальное значение числа $123$.
Глеб имеет возможность в~любой момент изменять порядок цифр числа, находящегося
на~экране.
Может~ли он~добиться того, чтобы число никогда не~стало четырёхзначным?

%\item
%Обозначим через $S(n)$ произведение ненулевых цифр числа $n$.
%Чему равно $S(1) + S(2) + \ldots + S(10000)$?

%\item
%Докажите, что предпоследняя цифра любой степени тройки всегда четна.

\item
Натуральные числа $a < b < c$ таковы, что $b + a$ делится на~$b - a$,
а~$c + b$ делится на~$c - b$.
Число $a$ записывается $2011$ цифрами, а~число $b$~--- $2012$ цифрами.
Сколько цифр в~числе $c$?

%%%%%%%%%%%%%%%%%%%%%%%%%%%%%%%%%%%%%%%%

\item
Существует~ли степень двойки, из~которой перестановкой цифр можно получить
другую степень двойки?

\item
Обозначим через $S(x)$ сумму цифр натурального числа $x$.
Решить уравнение
\[
    x + S(x) + S(S(x)) + S(S(S(x))) = 1993
\,.\]
%ММО 1993, 8.1 б)*.

\item
Докажите, что для любого натурального $n$ существует делящееся на~него число,
состоящее только ровно из~$n$ единиц и~некоторого количества нулей.

%\item
Обозначим $S(n)$ сумму цифр числа~$n$.
Пусть $a_n$~--- такая последовательность чисел, что $a_{k+1} = S(a_k)$ для
любого $k \geq 0$.
Найдите $a_6$, если $a_0 = 2^{1000000}$.

\item
Обозначим через $S(m)$ сумму цифр натурального числа~$m$.
Докажите, что существует бесконечно много натуральных~$n$ таких, что
\[
    S(3^n) \geq S(3^{n+1})
\,.\]
%Окружной этап всероса 1996-1997, 11.3.

%%%%%%%%%%%%%%%%%%%%%%%%%%%%%%%%%%%%%%%%

%\item
%Докажите, что первые цифры числа $2^{2^n}$ образуют непериодическую
%последовательность.

\item
Глеб написал на~доске ненулевую цифру и~приписывает к~ней справа по~одной
ненулевой цифре, пока не~выпишет миллион цифр.
Докажите, что на~доске не~более $100$ раз был написан точный квадрат. 

\end{problems}

