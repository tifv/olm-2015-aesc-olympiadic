% $date: 2014-11-12

\section*{Проверка связи по теории чисел}

\begin{problems}

\item
Докажите, что $91! \cdot 1901! - 1$ делится на~$1993$.

\item
Докажите, что любой простой делитель числа $2^{p} - 1$ имеет вид $2 k p + 1$.

\item
Пусть $p$ и~$q$~--- простые числа.
Докажите, что $p^q + q^p \equiv p + q \pmod{pq}$.

\item
Пусть $p$ и~$q$~--- последовательные нечетные числа.
Докажите, что $p^p + q^q$ делится на~$p + q$.

\item
Для данного простого $p$ рассмотрим число, в~котором сначала идут $p$~единиц,
потом $p$~двоек, $\ldots$, потом $p$~девяток.
Докажите, что оно сравнимо с~$123456789$ по~модулю $p$.

\item
Докажите, что для простого $p$ длина периода десятичной дроби $\frac{1}{p}$
является делителем числа~$(p - 1)$.

\item
Докажите, что $3^{2^n} - 1$
\quad
\sp делится на~$2^{n + 2}$
\quad
\sp не~делится на~$2^{n + 3}$.

\item
Можно~ли среди чисел $\frac{100}{1}$, $\frac{99}{2}$, $\ldots$, $\frac{1}{100}$
выбрать пять, произведение которых равнялось~бы единице?

\item
Через $\sigma(n)$ обозначим сумму делителей числа~$n$.
Докажите, что
\[
    \sigma(1) + \sigma(2) + \ldots + \sigma(n)
\leq
    n^2
\]

%\item
%Найдите все такие простые $p$ и~$q$, что $2^p + 1$ делится на~$q$ и~$2^q + 1$
%делится на~$p$.

%\item
%Пусть $p$ и~$q$~--- простые числа, $q > 5$.
%Известно, что $q \mid 2^p + 3^p$.
%Докажите, что $q > p$. 

\item
Докажите, что $(2 p - 1)! - p$ делится на~$p^2$.

\item
Решите уравнение $\phi(n) = \frac{n}{3}$.

\item
Найдите сумму всех правильных несократимых дробей со~знаменателем $n$.

\end{problems}

