% $date: 2014-11-21

\section*{Выпуклые оболочки}

\definition
Фигура называется \emph{выпуклой}, если для любых двух её~точек отрезок,
соединяющий эти точки, целиком принадлежит фигуре.

\definition
\emph{Выпуклой оболочкой} фигуры называется наименьшее выпуклое множество,
содержащее данную фигуру (в~качестве фигуры может выступать конечное множество
точек).

\statement
Выпуклой оболочкой нескольких точек на~плоскости и~нескольких произвольных
многоугольников является выпуклый многоугольник.

\begin{problems}

%\item
%Докажите, что любой выпуклый многоугольник площади $1$ можно поместить
%в~прямоугольник площади $2$.

\item
На~плоскости дано $5$ точек, причем никакие три из~них не~лежат на~одной
прямой.
Докажите, что четыре из~этих точек расположены в~вершинах выпуклого
многоугольника.

\item
На~плоскости дано несколько правильных $n$-угольников.
Докажите, что выпуклая оболочка их~вершин имеет не~менее $n$ углов.

\item
На~плоскости дано $n > 4$ точек.
Известно, что любые $4$ из~них являются вершинами выпуклого четырехугольника.
Докажите, что эти $n$ точек являются вершинами выпуклого n-угольника.

\item
\sp
Докажите, что у~любого многоугольника есть хотя бы одна диагональ, целиком
лежащая внутри.
\\
\sp
Какое наименьшее число таких диагоналей может иметь $n$-угольник?

\item
На~плоскости дано $n \geq 4$ точек, причем никакие три из~них не~лежат на~одной
прямой.
Докажите, что если для любых трех из~них найдется четвертая (тоже из~данных),
с~которой они образуют вершины параллелограмма, то~$n = 4$.

\item
На~плоскости дано конечное число точек.
Докажите, что из~них всегда можно выбрать точку, для которой ближайшими к~ней
являются не~более трех данных точек.

\end{problems}

