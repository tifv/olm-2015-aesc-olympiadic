% $date: 2014-10-08

\section*{Двойной подсчет}

% $authors:
% - Глеб Погудин

\begin{problems}

\item
Рассмотрим натуральные числа
$a_1 \leq a_2 \leq \ldots \leq a_n = m$.
Через $b_k$ обозначим количество таких $a_i$, что $a_i \geq k$.
Докажите, что:
\[
    a_1 + \ldots + a_n = b_1 + \ldots + b_m
.\]

\item
Будем рассматривать перестановки множества $\{ 1, \ldots, n\}$.
Через $p_n(k)$ обозначим количество тех из~них, что оставляют на~месте ровно
$k$ элементов.
Докажите, что
\(
    \sum\limits_{k = 0}^n kp_n(k) = n!
\).

\item
В~графе степень каждой вершины равна~$k$, у~каждой пары смежных вершин
ровно $l$ общих соседей, у~каждой пары несмежных~--- ровно $m$ общих соседей.
Докажите, что
\[
    m (n - k) - k (k - l) + k - m = 0
\]
% модельная задача про подсчет пар ребер, многие из~оставшихся через неё
% решаются

\item
Талантливые ученики в~количестве $n$~штук ($n \geq 3$ и~нечетно)
пишут тест из~$m$~вопросов
(каждый вопрос подразумевает ответ <<да>> или <<нет>>).
Известно, что у~любых двух совпадает не~более $k$ ответов
(каждый ответил на~все вопросы).
Докажите, что $\frac{k}{m} \geqslant \frac{n - 1}{2n}$.

\item
На~плоскости отмечено $n$~точек общего положения.
Известно, что для каждой из~отмеченных этих точек найдется $k$~отмеченных
точек, равноудаленных от~данной.
Докажите, что $k < \frac{1}{2} + \sqrt{2n}$.

% в добавку:

%\item
%На~олимпиаде было предложено $6$ задач.
%Известно, что для всякой пары задач больше $\frac{2}{5}$ участников решило обе
%эти задачи, и~никто не~решил все шесть.
%Докажите, что хотя~бы двое решило по~пять задач.

%\item
%В~галактическом парламенте есть две фракции~--- Пузатые и~Усатые, по~$2014$
%членов в~каждой.
%Кроме того, в~парламенте организован ряд комитетов.
%Известно, что для любого усатого и~любого пузатого найдется комитет, в~котором
%они состоят, и~каждый депутат состоит не~более чем в~$100$ комитетах.
%Докажите, что найдется комитет, среди членов которого хотя~бы $11$ пузатых
%и~$11$ усатых.

%\item
%В~таблице $m$ строк и~$n$ столбцов ($n > m$).
%В~некоторых клетках стоят звездочки, причем в~каждом столбце есть хотя~бы
%по~одной звездочке.
%Докажите, что найдется звездочка, с~которой в~одной строке больше звездочек,
%чем в~столбце.
%% классический гроб.
%% Хочется дать, чтобы никто не решил, чтобы разобрать, чтобы знали

\end{problems}

