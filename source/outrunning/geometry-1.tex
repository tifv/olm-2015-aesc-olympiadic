% $date: 2014-09-26

\section*{Геометрия-1}

% $authors:
% - Глеб Погудин

% $build$matter[print]: [[.], [.]]

В~задачах считаем, что треугольник $ABC$~--- остроугольный неравнобедренный,
если не~сказано обратное.
Точки $H$, $O$, $I$~--- ортоцентр, центр описанной и~вписанной окружностей
треугольника $ABC$ соответственно.

\begin{problems}

\item
\sp
Докажите, что точка $H$, отраженная относительно стороны $BC$, попадает в~точку
на~описанной окружности треугольника $ABC$.
\\
\sp
Докажите, что точка $H$, отраженная относительно середины $BC$, попадает
в~точку на~описанной окружности треугольника $ABC$, причем диаметрально
противоположную $A$.

\item
Обозначим через $B_1$ и~$C_1$~--- основания высот треугольника $ABC$, опущенных
из~вершин $B$ и~$C$ соответственно.
Пусть описанные окружности треугольника $ABC$ и~треугольника $A B_1 C_1$
пересекаются в~точке~$K$, отличной от~$A$.
Докажите, что прямая~$KH$ делит сторону~$BC$ пополам.

\item
Точки $K$, $L$, $M$ и $N$~--- середины сторон соответственно
$AB$, $BC$, $CD$ и~$DA$ вписанного четырёхугольника $ABCD$.
Докажите, что ортоцентры треугольников $AKN$, $BKL$, $CLM$ и~$DMN$ являются
вершинами параллелограмма.

\item
$B_1$ и~$C_1$~--- основания высот треугольника $ABC$, опущенных из~вершин $B$
и~$C$ соответственно.
Обозначим точку пересечения отрезков $B_1 C_1$ и~$AH$ за~$P$, а~точку
пересечения прямых $AO$ и~$BC$ за~$Q$.
Пусть $M$~--- середина $BC$.
Докажите, что прямые $HM$ и~$PQ$ параллельны.

\end{problems}

\rule[0.5ex]{\textwidth}{0.5pt}

\begin{problems}

\item
Докажите, что основания перпендикуляров, опущенных из~вершины~$A$
на~биссектрисы внутренних и~внешних углов $B$ и~$C$, лежат на~средней линии
треугольника $ABC$, параллельной $BC$.

\item
Вписанная окружность треугольника $ABC$ касается сторон $AB$ и~$AC$ в~точках
$K$ и~$L$ соответственно.
Пусть точки $M$ и~$N$~--- середины $AB$ и~$BC$ соответственно.
Докажите, что прямые $CI$, $KL$ и~$MN$ пересекаются в~одной точке.

\item
В~треугольнике $ABC$ ($AB > BC$) проведены биссектриса~$BL$ и~медиана~$BM$.
Прямая, проходящая через $M$ параллельно стороне~$AB$, пересекает $BL$
в~точке~$D$.
Прямая, проходящая через $L$ параллельно стороне $BC$, пересекает $BM$
в~точке~$E$.
Докажите, что $ED \perp BD$.

\item
$ABC$~--- равнобедренный треугольник ($AB = BC$).
Средняя линия, параллельная $BC$, пересекает вписанную окружность треугольника
$ABC$ в~точке~$K$, не~лежащей на~стороне~$AC$.
Докажите, что касательная к~вписанной окружности в~точке $K$, сторона~$AB$
и~биссектриса угла~$C$ пересекаются в~одной точке.

\end{problems}

