% $date: 2014-11-19

\section*{Принцип крайнего как лекарство от комбинаторной геометрии}

\begin{problems}

\item
Из~точки $O$ выходит несколько лучей.
Угол между любыми двумя меньше $120^{\circ}$.
Докажите, что найдутся два луча такие, что все остальные содержатся в~угле
между ними.

\item
Несколько прямых общего положения разбивают плоскость на~части.
Докажите, что хотя~бы одна из~этих частей~--- угол.

\item
Длина наибольшей стороны треугольника равна $1$.
Докажите, что три круга радиуса $\frac{1}{\sqrt{3}}$ с~центрами в~вершинах
покрывают треугольник целиком.

\item
Докажите, что из~любых пяти точек общего положения можно выбрать четыре,
являющиеся вершинами выпуклого четырехугольника.

\item
Существует~ли такой выпуклый пятиугольник $ABCDE$, что все углы
$ABD$, $BCE$, $CDA$, $DEB$ и~$EAC$~--- тупые?

\item
Докажите, что для любой точки $O$ внутри выпуклого многоугольника найдется
сторона $l$ такая, что проекция $O$ на~прямую, содержащую $l$, лежит на~$l$.

\item
Докажите, что в~любом многоугольнике можно провести несколько диагоналей,
пересекающихся только в вершинах, которые разбили бы многоугольник
на~треугольники.

\item
Конечное множество точек на плоскости удовлетворяет следующему условию:
для любых двух точек множества на прямой, их соединяющей, найдется третья точка
из множества.
Докажите, что все точки множества лежат на одной прямой.

\end{problems}

\iffalse % BEGIN НА СВЕТЛОЕ БУДУЩЕЕ

\item
Вершины выпуклого $n$-угольника ($n > 4$) расположены в~узлах сетки.
Докажите, что внутри многоугольника есть хотя~бы одна целая точка.

\item
Фигура $S$ на~плоскости имеет площадь больше $1$.
Докажите, что найдутся такие точки $A, B \in S$, что вектор $\overline{AB}$
имеет целочисленные координаты.

\item
Внутри выпуклого стоугольника отмечено $k$ точек ($2 \leq k \leq 50$).
Докажите, что можно выбрать $2k$ вершин стоугольника так, что $2k$-угольник
с~вершинами них содержит все отмеченные точки.

\fi % END НА СВЕТЛОЕ БУДУЩЕЕ

